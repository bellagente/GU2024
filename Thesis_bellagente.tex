\documentclass[12pt,a4paper,openany]{book} %se vuoi puoi anche mettere tra le quadre openright,dovrebbe essere automatico per il book
\usepackage[english]{babel}
\usepackage[T1]{fontenc}
\usepackage{graphicx}
\usepackage{caption}
\usepackage{amsmath}
\usepackage{amssymb} 
\usepackage[binding= 0.6cm]{layaureo} %per migliorare impaginamento
\usepackage[hidelinks]{hyperref} %collegare indici a testo e riferimenti alla bibliografia: put [hidelinks] before {} to make them invisible
\usepackage{caption}
\captionsetup{figureposition=bottom,font=footnotesize}
\usepackage{booktabs} 
\captionsetup{tableposition=top,font=footnotesize}
\usepackage{color}
\usepackage{fancyhdr}
\usepackage{emptypage}


\usepackage[round, sectionbib]{natbib}
\bibliographystyle{apalike-ejor}





\usepackage{titlesec}
\titleformat{\chapter}[hang] 
{\normalfont\huge\bfseries}{\chaptertitlename\ \thechapter:}{0.5em}{} 

\usepackage{titlesec} % to format section headings
\setcounter{secnumdepth}{4}
\setcounter{tocdepth}{4}
\titleformat{\subsubsubsection} % Changing subsubsubsection
{\normalfont\normalsize\bfseries}{\thesubsubsubsection}{1em}{}
\titlespacing*{\subsubsubsection} {0pt}{3.25ex plus 1ex minus .2ex}{1.5ex plus .2ex}



%%%%%%%%%%%%%%%%%%%%%%%%%%%%%%%%%%
\begin{document}

%%%%%%%%%%% Frontespizio%%%%%%%%%%%%%%%%%%%%%%%%%%%%%%%%%%%%%%%%
\begin{titlepage}
	\thispagestyle{empty}
	
	\begin{center}
		\vskip 0.7cm 
		\includegraphics[width=8cm]{./intro/gbg}
		%\vskip 0.5cm 
		
		\large
			%\text {\bf UNIVERSITY OF GOTHENBURG}\\
			\text {\bf  Department of Earth Sciences}\\
			
			\vskip 0.01cm 
		\large \bf
			M.Sc. Atmosphere, Climate and Ecosystems \\  Master thesis 45 hp
		
		\vskip 3.5cm
		\LARGE
			\textbf{Effects of cloud microphysics on precipitation simulation over the \\Tibetan Plateau}
		\vskip 3cm
		
		\normalsize
			\begin{minipage}[b]{10.4cm}
			         Student:\text{~Irene Elisa Bellagente}\\ \\
                               % \vskip 0.5cm 
                                Supervisor:\text{~Hui-Wen Lai}\\ \\
                                Examiner:\text{~Deliang Chen}\\ \\
				%Matricola: xxxxxx/y
			\end{minipage}

			
			\vskip 4.5cm
			Gothenburg, Sweden, June 2024
	\end{center}
	
	\vfill
	\eject
\end{titlepage}
%%%%%%%%%%%%%%%%%%%%%%%%%%%%%%%%%%%%%%%%%%%%%%%%%%%%%%%%%%%
%%%%%%%%%%%%%%%%%%%%%%%%%%%%%%%%%%%%%%%%%%%%%%%%%%

%questo è per cambiare i titoletti in alto nelle pagine di testo
\pagestyle{fancy}
\fancyhead{} % cancella tutti i campi
%\rhead[\textsl{CAPITOLO 1.~~ FONDAMENTI FISICI}]{}\lhead[]{\textsl{1.1.~~PROPAGAZIONE DI RADIAZIONE E.M.}} 
\renewcommand{\headrulewidth}{0pt}
%%%%%%%%%%%%%%%%%%\usepackage[colorlinks]{hyperref}
\pagenumbering{roman}
%\tableofcontents 
%\mainmatter %questo è per avere nell'indice i numeri, altrimenti mette i numeri romani
%\pagenumbering{arabic}
%\setcounter{page}{1}

\chapter*{Abstract}
\addcontentsline{toc}{chapter}{Abstract}
The Tibetan Plateau (TP) is the tallest and largest plateau in the world, often referred to as the Third Pole as it is the third largest freshwater storage on Earth providing water for nearly two billion people. Climate change is severely altering its energy budget and water cycle, but future projections contain major uncertainties due to serious challenges in the simulations of weather and climate. One of the main contributors to modelling uncertainties are cloud feedbacks and related processes, especially regarding cloud microphysics (MP). In climate models, cloud MP is parametrized by MP schemes, which simulate the formation of hydrometeors and their fall to the surface as precipitation. The aim of this thesis is to study how precipitation and related processes are represented by different MP schemes, and identifiy the main sources of uncertainties in precipitation simulations over the TP. Specifically, we focused on two case studies of heavy precipitation - one mesoscale convective system (MCS) and one heavy snow system - and six MP schemes within the Weather Research and Forecasting model (WRF). Our results show that temporal and spatial patterns of precipitation are subjected to major differences in intensity and timing depending on the choice of MP scheme. The simulation of factors involved in precipitation formation, like the energy budget, moisture content and convection, has been found to substantially affect the precipitation outcome. It has been observed that vertical velocity (W) is in agreement with precipitation patterns and can be considered as a major source of discrepancies in precipitation simulations. Other factors, such as horizontal winds and cloud composition, have been analysed and linked to alterations in precipitation location and intensity. Hydrometeors and their representation in MP schemes, especially when it comes to solid-phase water species, have been identified as crucial sources of uncertainties. In conclusion, the observed variations in precipitation simulations due to the choice of MP scheme highlight the necessity of improvement of cloud MP processes parametrization in climate models.\\ \\
\textbf{Key words:} \textit{Tibetan Plateau, WRF, convection-permitting simulations, cloud microphysics, precipitation.}\\

\chapter*{Sammanfattning}
\addcontentsline{toc}{chapter}{Sammanfattning}
Tibetanska platån (TP) är den högsta och största platån i världen, ofta kallad den tredje polen eftersom den är den tredje största färskvattenreserven på jorden och förser nästan två miljarder människor med vatten. Klimatförändringar förändrar kraftigt dess energibudget och vattenkretslopp, men framtida prognoser riskerar att bli osäkra på grund av allvarliga utmaningar i simuleringar av väder och klimat. En av de främsta faktorerna till modelleringsosäkerheter är molnåterkopplingar och relaterade processer, särskilt när det gäller molnmikrofysik (MP). I klimatmodeller parametriseras moln-MP av MP-system, som simulerar bildandet av hydrometeorer och deras fall till ytan som nederbörd. Syftet med denna avhandling är att studera hur nederbörd och atmosfäriska processer representeras av olika MP-system och identifiera de främsta källorna till osäkerheter i nederbördssimuleringar över TP. Specifikt fokuserade vi på två fallstudier av kraftig nederbörd – ett mesoskaligt konvektivt system (MCS) och ett kraftigt snöfallssystem – och sex MP-system inom väder- och forskningsprognosmodellen (WRF). Våra resultat visar att tidsliga och rumsliga mönster för nederbörd är föremål för stora skillnader i intensitet och timing beroende på valet av MP-system. Simuleringen av faktorer som är involverade i bildandet av nedebörd, såsom energibudget, fuktinnehåll och konvektion, har visat sig ha en betydande inverkan på nederbördens utfall. Det har observerats att vertikal hastighet (W) stämmer överens med nederbördsmönster och kan betraktas som en huvudkälla till skillnader i nederbördssimuleringar. Andra faktorer, såsom horisontella vindar och molnsammansättning, har analyserats och kopplats till förändringar i nederbördens plats och intensitet. Hydrometeorer och deras representation i MP-system, särskilt när det gäller fasta vattenarter, har identifierats som avgörande källor till osäkerheter. Sammanfattningsvis belyser de observerade variationerna i nederbördssimuleringar på grund av valet av MP-system behovet av förbättringar av parametriseringen av moln-MP-processer i klimatmodeller.\\ \\
\textbf{Nyckelord:} \textit{Tibetanska platån, WRF, konvektion-stillåtande simuleringar, molnfysik, nederbörd.}


\tableofcontents 
\mainmatter %questo è per avere nell'indice i numeri, altrimenti mette i numeri romani
\pagenumbering{arabic}
\setcounter{page}{1}


\chapter{Introduction}
\pagestyle{fancy}
\fancyhead{} % cancella tutti i campi
%\rhead[\textsl{Chapter 1.~~ Introduction}]{}
\rhead{\textsl{Chapter 1: Introduction}}
\renewcommand{\headrulewidth}{0pt}
The Tibetan Plateau (TP) is the tallest and largest plateau in the world, with an average elevation of more than 4,000 m and an area of about 2.5 $\times$ $10^6$ km$^2$  \citep{Gao2018} (Fig. \ref{TP}). It is often referred to as the Third Pole, because altogether with the surrounding mountains it is the third largest freshwater storage on Earth after the Arctic and Antarctica \citep{Kukulies2023}. Glaciers in the Himalayas feed ten of the world’s most important river systems and supply the basic needs of water of nearly two billion people. However, they are regarded as some of the most vulnerable ‘water towers’ in the world \citep{douville2021,Immerzeel2019}. \vskip0.2cm

Climate change is severely altering the energy budget and water cycle of the TP, but future precipitation projections in this region contain major uncertainties due to serious challenges in the simulation of weather and climate processes in this topographically complex area \citep{Prein2022}. On top of the issues caused by the peculiar topography, cloud feedbacks add another layer of complexity which is indipendent from the study area. In fact, the response of clouds to climate change is spatially heterogeneous, microphysically complex and highly model dependent. Cloud feedbacks are a major contributor to uncertainties in climate simulations all over the world \citep{forster2021,RaviKiran2015,Yin2022}. \vskip0.2cm

For all the aforementioned reasons, the present work aspires to contribute towards broadening our knowledge on the simulation of precipitation events and associated cloud microphysics processes over the TP. Our analysis comprises the study of the differences in temporal evolution and spatial patterns of precipitation as well as the study of the energy budget, atmospheric instability, moisture transport and hydrometeors distributions. We also identify the key variables leading to modelling uncertainties and highlight which factors are poorly simulated. The results can be useful towards the improvement of precipitation and cloud microphysics simulations. \vskip0.2cm
\begin{figure}[t!]\centering\includegraphics[width=1\textwidth]{./intro/TP}\caption{Topography of the TP. The black contours indicate elevation at 3000 m.}\label{TP}\end{figure}
In this chapter we will introduce an overview of the study area, the most relevant physical mechanisms related to precipitation and some basic concepts of climate modelling. The goal is to contextualise our study within a broader climate change perspective and provide necessary information to fully understand the upcoming sections. In Chapter 2 a detailed rendition of the data and methods used will be given, including a comprehensive description of the precipitation events and the microphysics schemes studied. In Chapter 3 we will illustrate the results of our analysis, which will be discussed and compared with existing literature in Chapter 4. Lastly, Chapter 5 contains a short summary and the final conclusions.


\section{The Tibetan Plateau's climate}
The TP region is located in southeast China between latitudes 25-40°N and longitudes 75-115°E and it has one of the most complex climates in the world. It is characterized by wet and humid summers with cool and dry winters. Roughly 60–90\% of the total annual precipitation occurs during summer, while only about 10\% falls during winter. The mean annual precipitation ranges from more than 800 mm in the southeastern part to below 100 mm in the northwestern side \citep{Xu2007}. Average temperatures stay around 15°C during summer while they drop below -10°C during winter \citep{Deng2017}.\vskip 0.2cm

Precipitation occurs when moist air reaches a height where temperature is cold enough for condensation of water vapor onto cloud condensation nuclei or ice nuclei. Over the TP, precipitation is influenced by a number of factors, including surface heating, moisture supply, large-scale atmospheric flows and orographic lifting. During summer, since the plateau is warmer than the surrounding atmosphere, the TP becomes a heat source and prompts a wide range of upward air motion, triggering circulations that are crucial for the regional climate and weather. On the other hand, over the southern Himalayas a lot of convective precipitation is induced by strong uplifting caused by the deep slopes of the mountains \citep{Kukulies2023}. \vskip 0.2cm

On the plateau, temperature and precipitation are characterised by strong diurnal and seasonal variations due respectively to solar radiation and large scale atmospheric circulations \citep{Xu2007}. In fact, over the southern and eastern TP summer precipitation is strongly associated with the Asian summer monsoons, while in the western TP it is mainly related to westerly jets \citep{Lai2021,Lai2023a}. The TP is also an active region of mesoscale convective systems (MCSs) accompanied by frequent mesoscale cyclonic low pressure systems called plateau vortexes. In summer months, MCSs are more frequent but less intense and smaller in size than those over the ocean and other land regions. The peculiar topography, atmospheric circulation and thermodynamics cause unique weather and climate systems. As a result, substantial differences in the microphysical characteristic of convective clouds and precipitation arise when compared to other regions \citep{Gao2018}. \vskip 0.2cm

Over the TP we can identify four major cloud types, varying in frequency and spatial distribution depending on the season. Cirrus clouds prevail as the most common and widespread cloud type over the whole year, followed by altocumulus clouds that are mainly distributed in bands along the west and south sides of the TP. The third most frequent type are altostratus clouds, which are primarily found in the Sichuan Basin and occasionally extending into the altocumulus band in the southern region of the TP. Lastly there are deep convective clouds, which are observed over the southern part of the TP during the warm season and contribute significantly to the total precipitation in spite of their relatively low frequency \citep{Cai2022}. 
% cirrus: thin and high, ice and warming (10 km from ground) - altocumulus: layered and mid level, water and cooling/warming depends (5km) - altostratus: sheet and mid level, water/ice and warming/cooling depends (5km) - deep convective clouds/cumulonimbus: towering and height varies a lot (500 m to 20km), ice/water/hail and cooling/warming depends (thickness: 2 to 10+ km) 


\section{Scientific background}
\subsection{Precipitation formation mechanisms} %Cloud microphysics includes the formation of clouds and precipitation as well as its distribution and intensity.
Cloud microphysical processes are known to play a major role in precipitation formation and evolution, and local atmospheric thermodynamics \citep{Hazra2017}.  As air rises, water vapour condenses into small water droplets or ice particles, whose formation and growth mechanisms are called cloud microphysics (MP). Based on temperature, clouds can be categorised into warm and cold clouds. In the first case temperature is above 0°C and the cloud is composed by liquid particles, while in the second case temperatures drop below -40°C and the cloud contains only ice particles. When temperatures are between -40°C and 0°C, the cloud may be made of a combination of pure ice and liquid particles, known as mixed-phase. Moreover, when temperatures are below 0°C the liquid water cloud is considered as supercooled \citep{Wang2021}. \vskip0.2cm

In warm clouds, the initial formation of liquid drops typically occurs as molecules condense onto a cloud condensation nucleus (CCN), and their growth proceeds until they evolve into cloud droplets and eventually precipitation particles \citep{Pruppacher2010}. This process involves nucleation, wherein water vapor molecules in the air collide and coalesce to generate liquid-phase drops. For cloud droplets to form, the air must be supersaturated, with relative humidity exceeding 100\%. This mechanism is called homogeneous nucleation, to be distinguished from heterogeneous nucleation where molecules aggregate onto foreign substances known as aerosol particles. The physics of the nucleation process indicates that the first droplets in a cloud will tend to form around the largest and most soluble CCN. Consequently, the sizes and compositions of aerosol particles within a given air sample significantly influence the size distribution of particles nucleated within a cloud. Once formed, water droplets may further grow by condensation as vapor diffuses towards them. Coalescence is another relevant growth mechanism through which cloud drops merge together to form larger particles. When cloud droplets become big enough (usually over 100 $\mu$m), they fall as precipitation particles due to the gravitational force prevailing \citep{Houze2014}. \vskip 0.2cm

In cold clouds, homogeneous nucleation of ice particles occurs directly from the liquid phase \citep{Pruppacher2010}. This process typically begins around -40°C, however the exact temperature depends on the ambient supersaturation. Similarly to drop formation, the efficacy of the nucleation is influenced by water vapor content but also temperature, as larger drops freeze at slightly higher temperatures than smaller ones. Ice crystals are present at temperature between 0°C and about -38°C, but since homogeneous nucleation does not occur in this temperature range, the crystals must form via heterogeneous process. The primary issue with this mechanism is that molecules in the ice phase are arranged in a highly ordered crystal lattice. For the nucleation to happen, the substrates acting as CCN must have a lattice structure similar to that of ice. Various methods can trigger the formation of an ice crystal from an ice nucleus, including condensation nucleation, immersion freezing, contact nucleation, and deposition nucleation. Once the ice particle is formed, its growth can occurr by deposition as ambient vapor diffuses towards the particle. Other relevant growth mechanisms include aggregation, where ice particles collect other ice particles, and riming, where ice particles collect liquid drops which freeze on contact. Ice particles can either fall in solid form or melt into liquid water when they come into contact with air or water that is above 0°C \citep{Houze2014}. \vskip 0.2cm
%If the CCN on which the drop forms is the ice nucleus, the process is referred to as condensation nucleation. If the nucleation is caused by any other nucleus suspended in supercooled water, the process is called immersion freezing. Another process is contact nucleation, in which drops may also be frozen if an ice nucleus in the air comes into contact with the drop. Lastly, the ice may be formed on a nucleus directly from the vapor phase by deposition nucleation.
Over the TP, ice and mixed-phase clouds produce the vast majority of precipitation as liquid water clouds are not so common. In particular, in the central part of the TP, precipitation is strongly affected by ice clouds, while in the southeastern region during the warm season it is mostly dominated by mixed-phase clouds \citep{Wang2021}.

\subsection{Stratiform and convective cloud systems} 
Precipitation can be distinguished into two types: stratiform, falling from nimbostratus clouds, and convective, generated from cumulus and cumulunimbus clouds. In the first case there are broad sheet-like clouds with no significant vertical air movement, while in the latter case dense clouds are characterised by significant vertical extent and intense upward currents. These low cloud types, typically found below 2 km in altitude, may appear independently or intertwined within the same atmospheric system \citep{Houze2014}. \vskip 0.2cm

Warm stratiform clouds with tops below the 0°C level can produce precipitation, but the majority of stratiform precipitation falls from nimbostratus clouds extending well above the 0°C level and containing ice particles.  In deep nimbostratus clouds, precipitation largely depends on the growth of ice hydrometeors while slowly descending. Precipitation particles, initially formed as ice particles in the upper parts of the cloud, undergo melting as they fall due to rising temperatures. The higher the altitude at which the ice particles are formed, the longer they will be able to grow by deposition of vapor, aggregation and riming. The time available for growth of the ice particles falling from cloud top is $\sim$1–3 hours, which is the time it takes a particle falling at $\sim$1–3 m/s to descend 10 km. Aggregation, occurring predominantly within 1 km of the 0°C level, redistributes precipitation mass into fewer but larger particles, leading to the formation of large raindrops upon melting \citep{Houze2014}. \vskip 0.2cm

Conversely, in convective precipitation the mean vertical air velocity, typically ranging from 1 to 10 m/s, exceeds the fall speeds of ice crystals and snow. Because the time for the growth of precipitation particles is very limited, typically around half an hour, the initial precipitation particles must originate not far above the cloud base. The growth process can start with the formation of the cloud, since updrafts are strong enough to carry the growing particles upward until they become heavy enough and begin to fall towards the ground. In this case, the only microphysical growth mechanism rapid enough is accretion of liquid water. Unlike in stratiform precipitation, where vapor deposition and ice particle aggregation dominate, convective precipitation relies heavily on liquid water accretion. As convective cloud elements dissipate, the precipitation particles left aloft adopt a stratiform character, sometimes forming large regions of stratiform precipitation when clusters of convective cells weaken in a given area \citep{Houze2014,Pruppacher2010}. \vskip 0.2cm

The TP experiences intense summer convection, making its downstream regions global hotspots for organized convection over land. This is influenced by midlatitude atmospheric dynamics, moisture influx from the tropic, and the impact of mountain chains \citep{Kukulies2023,Curio2019}. MCSs are crucial components of this climate, driven by monsoon circulation, surface heating and moisture supply. These systems predominantly occur during summer, mainly over the central and eastern TP and downstream regions. MCSs are characterized by cloud shields that can vary significantly in size but typically have minimum horizontal extents of around 100 km. Globally, MCSs are known for their efficiency in producing rainfall due to prolonged heavy precipitation and extensive areas of associated stratiform precipitation. Over the TP, MCSs contribute approximately to 20\% of the total precipitation, mostly in spring and summer \citep{Feng2021,Kukulies2021}.


\subsection{Cloud phases and radiative effects}
The presence of clouds can significantly affect the increase or decrease of Earth's net radiation. This effect is closely related to the cloud type and composition as there are differences between liquid and ice particles in refractive indices, sizes, concentration and shapes. Thus, depending on their phases, clouds have a distinct radiative effect \citep{Wang2021,Glazer2018}. Globally, clouds in the lower atmosphere contribute the most to the net energy balance of the Earth \citep{forster2021}. For example, a low-level water cloud can have a strong negative net radiative effect at the surface, resulting in a cooling effect. In contrast, thin cirrus clouds cause a positive radiative effect at the surface and in the atmosphere, causing an increase in temperature. For mixed-phase clouds, the radiative effect is closely associated to the partitioning between liquid and ice phases \citep{Wang2021}.  Due to smaller particle sizes, the radiation effect of a liquid cloud is much greater than that of an ice cloud with the same water content. The global radiative forcing related to ice clouds causes the net warming of the Earth system, while liquid-phase clouds have a negative net radiative effect. Therefore, accurately understanding the spatial and temporal distribution characteristics of cloud types and phases is crucial for studying their impact on Earth's radiation balance \citep{Cai2022}. \vskip 0.2cm

Over the TP, total cloud coverage is higher during summer than in winter. Ice clouds dominate the cloud coverage during winter, constituting approximately 84\% of the total, while warm clouds play a minor role. Mixed-phase clouds are significant, representing around 50\% of the total coverage during summer, particularly in the southeastern region. Overall, in the TP region the total cloud coverage has a negative radiative effect, which is relatively weak during winter and intensifies during summer \citep{Wang2021}.
%the total radiative effect is negative because, despite ice clouds (causing warming) are the majority, the 15% mixed-phase/liquid clouds have a bigger effect. therefore the total result is negative (cooling). In fact, the negative radiatve effect intensifies in summer when the mixed-phase clouds represent 50% of the total cloud coverage.


\section{Climate change}
\subsection{Global and regional effects on the hydrological cycle}
Variations of the Earth’s energy budget caused by anthropogenic radiative forcings lead to significant changes in the global water cycle, which in turn heavily affect terrestrial ecosystems and human societies. Changes in the amount and seasonality of precipitation caused by climate change have long been recognized by the Intergovernamental Panel on Climate Change \citep{douville2021}. \vskip 0.2cm

According to the Clausius-Clapeyron law, an increment in the water holding capacity of the low atmosphere of about 7\% per 1°C of warming explains a similar intensification of heavy precipitation \citep{seneviratne2021}. Nonetheless, the average global increase in precipitation is partly balanced out by quick atmospheric adjustments in response to heating caused by greenhouse gases and cooling caused by aerosols, and it is assessed to be around 2-3\% per 1°C of warming \citep{douville2021}. Thus, water vapour persist longer in the atmosphere and leads to alterations in precipitation intensity, frequency and duration. In the atmosphere, water is primarily found in the form of water vapour but it also occurs as ice and liquid droplets within clouds, substantially affecting the Earth’s energy balance and the water cycle. In fact, the additional latent heat released from the thermodynamic increases in moisture can enhance precipitation rates by causing dynamic responses within storms, namely by strengthening convective updrafts \citep{seneviratne2021}. There's strong evidence that thermodynamics leads also to an increase in convective available potential energy (CAPE) with warming and therefore enhances the intensity of convective storms.\vskip 0.2cm

At regional scales, the transport of moisture dominates water cycle changes, which depend not only on thermodynamic but also on dynamical processes, such as atmospheric circulation and topographic effects. Generally speaking, the magnitude of extreme wet and dry events increases with rising temperatures but atmospheric circulation patterns affect their location and frequency. The abundance or lack of water raises the likelihood of flooding or drought as well as their overall impact, as precipitation becomes more unpredictable in a warming climate \citep{douville2021}.\vskip 0.2cm

High mountain regions have undergone enhanced warming since the beginning of the $20^{th}$ century, resulting in decreased snowpack on average and glaciers retreating all over the globe. In mountain watersheds, snowfall represents a major component of precipitation as it is strongly linked with important mechanisms such as albedo feedbacks and seasonal runoff. Elevation dependent warming could cause speedy variations in the snowline, the glacier equilibrium altitude and the snow/rain transition height \citep{douville2021}. Changes in the cryosphere have affected the amount and seasonality of runoff in river basins dependent on snow and glaciers, with significant local impacts on water resources \citep{portner2019}. The melting of glaciers caused by rising temperatures can initially lead to larger runoff volumes but they will eventually reduce as most glaciers continue to shrink, potentially causing shortages of fresh water in the downstream regions \citep{douville2021}.

\subsection{Influences on clouds}
The alteration of cloud characteristics is related to several meteorological variables, including temperature, relative humidity, vertical velocity, divergence and aerosol concentrations \citep{Cai2022}. Human activities have impacted clouds properties in two main ways: by changing the amount of aerosol particles in the atmosphere and by warming the Earth’s surface. The atmospheric concentration of aerosols has markedly increased, causing cloud droplets to become more numerous and smaller and thus reflect more incoming energy. Hence, higher aerosols concentrations have had a cooling effect. However, it is anticipated that this cooling effect will diminish in the future as global air pollution policies progress, reducing aerosol emissions. On the other hand, warming has alterated certain cloud properties, such as their altitude, amount and composition, thereby affecting the Earth’s energy budget and temperature. Cloud feedback mechanisms could either amplify or offset future warming, yet climate projections remain uncertain due to serious challenges in accurately simulating cloud physics in models. The problem arises from the fact that clouds can change in many ways and that their processes occur on much smaller scales than global climate models can explicitly represent. Nonetheless, improved understanding of how clouds respond to warming has increased confidence that future changes will likely contribute to additional warming \citep{forster2021}.

\subsection{Effects on the Tibetan Plateau} %Moreover, region-wide snowfall during winter has increased but summer snowfall has become more rare during the last 50 years. / A previous study  concluded that the interaction between clouds and radiation is a primary factor in the warming of this region (Liu 2009 -to be d)
In the TP evidence suggests that enhanced warming has occurred and it is especially amplified around 4000 m of altitude \citep{portner2019}. As a consequence, hot extremes are progressively becoming more frequent and intense whereas cold extremes occur more rarely and less intensive. A rise in daily precipitation extremes over the northern and the western Himalayas was observed in the last 70 years, while in the eastern part a decrease was experienced \citep{seneviratne2021}. The annual mean air temperature between 1980 and 2018 showed an overall warming trend compared to the two previous decades, indicating a rate of warming twice as fast as the global average (Fig.\ref{TPPT}a). Precipitation datasets over this area have still large uncertainties and many of them understimate the precipitation rates at high elevations. Nonetheless, annual mean precipitation has exhibited a clear increasing trend between 1980 and 2018, especially in the last 20 years (Fig.\ref{TPPT}d) \citep{Zhang2020}.
\begin{figure}[t!]\centering\includegraphics[width=1\textwidth]{./intro/TP-precip-temp}\caption{Temperature and precipitation changes. a) Temperature difference between 1998-2018 and 1980-1997. b) Annual temperature anomalies between 1980 and 2018 for the TP. c) Change in temperature for 1970-2016. d) Precipitation difference between 1998-2018 and 1980-1997. e) Anomaly of annual precipitation for 1980-2018. f) Change in precipitation for 1970-2016. The + indicate significant linear trends at a 95\% confidence level \citep{Zhang2020}. } \label{TPPT} \end{figure}

Across the Himalayan mountains, extreme rainfall events are projected to increase throughout the 21$^{st}$ century both in terms of frequency and intensity \citep{portner2019}. Multiple studies on extreme precipitation over western Himalayas observed substantially high moisture fluxes in the lower atmosphere from the Bay of Bengal and the Arabian Sea towards the western Himalayas during the events \citep{Karki2018}. The warm and moist air masses were topped by dry continental air originating from Afghanistan and the TP. The combination of this phenomena with a horizontal barrier due to the Himalayan mountains prevented the premature release of CAPE and caused a destabilization of atmosphere, which favoured the development of extreme convective precipitation. On the other hand, other studies showed the interaction of westerly and monsoonal air masses as a major synoptic feature of the extreme events. Over central Himalaya, a southward intrusion of westerly trough and the northward shift of the monsoon trough  were identified as the major drivers of extreme precipitation events. The majority of floods associated with monsoons are related to intense and thick convective systems. However, there is strong evidence of some cases of heavy stratiform precipitation, which indicate that the nature of high intensity local precipitation events over the Himalayas can vary \citep{Karki2018}. \vskip 0.2cm

Because of the TP’s complex terrain, ground-based and satellite observations are affected by major uncertainties, severely limiting the study of relevant local physical processes, systems and climate change induced alterations of the water cycle. The resolution of available datasets is not accurate enough to reproduce precipitation over complex topography or caused by convection and other mesoscale systems characterized by small scale variability. As a result, we have only a partial understanding of the impacts of climate change on the TP and downstream regions \citep{Prein2022}. \\


\section{Climate modelling}
\subsection{From global models to regional models}
Numerical weather prediction (NWP) models are essential to forecast high impact precipitation events and mitigate their effects. Despite significant advancement in this field in recent years, accurately simulating time, location and intensity is still a challenge over complex terrain due to the intricate interplay of synoptic, mesoscale and local scale mechanisms \citep{Gao2018}. The performance of the model is influenced by inaccuracies of terrain and land use representation, imperfect parameterizations and boundary conditions. Due to their global scale resolution and oversimplified physics schemes, global climate models (GCMs) are able to represent large scale precipitation patterns, but the local processes and phenomena that affect precipitation are not represented precisely enough \citep{Orr2017}. Most current GCMs do not yet simulate MCSs because convection in these models is parameterized without considering mesoscale processes. As a result, global and regional climate models with parameterized convection largely fail to reproduce major precipitation events \citep{Feng2018}. \vskip 0.2cm

Regional climate models (RCMs) can generally be used to simulate climate at much coarser spatial scales, thus better representing the interaction of large scale circulations, namely the winter midlatitude westerlies and summer monsoon, with regional precipitation mechanisms such as orographic lifting \citep{Orr2017}. Among the various mesoscale NWP models, the Weather Research and Forecasting (WRF) is one of the most popular and widely used state-of-the-art models. Over the Himalayas, several high resolution WRF simulations have shown an improved simulated spatial and temporal variations of precipitation due to more detailed representation of terrain, land use and the explicit representation of convection \citep{Karki2018}. In fact, recent advancements in computational resources have enabled RCM simulations to be run at convection permitting resolutions ($\le$4 km grid spacing), eliminating the need for parameterizations \citep{Podeti2020,Prein2022}. Compared to models with parameterized convection, convection permitting models (CPMs) produce more realistic timing and intensity of summertime precipitation and better represent high intensity precipitation events. CPMs provide the ability to simulate interactions between convection and large-scale circulations, which are especially important for MCSs as they strongly interact with their large scale environments \citep{Feng2018,Kukulies2023a}.

\subsection{Microphysics schemes} 
In climate models, MP processes are parametrized by MP schemes. They simulate the formation of various hydrometeors (i.e. cloud water, water vapor, cloud ice, snow, graupel and hail), their growth and fall to the surface as precipitation. They also reproduce the vertical distribution of different hydrometeors, the latent heat release and absorption, which affect the vertical velocity profile and thus the local scale atmospheric dynamics \citep{Karki2018}. Because the evolution of deep convective clouds is affected by small scale microphysical processes, their representation in models influences the realism of the simulation itself. The complexity of cloud MP schemes in atmospheric models is an attempt to reach a sufficient level of physical realism in the representation of cloud processes \citep{Furtado2018}. \vskip 0.2cm

A major challenge in clouds and precipitation events simulations is a lack of accuracy when it comes to describing cloud MP through parametrization schemes. Most of these schemes only consider relatively simple microphysical processes, such as direct division between cloud condensation and precipitation, and do not include a comprehensive conversion among different types of hydrometeors. Consequently, the simulation of microphysical cloud and precipitation responses to aerosol-related perturbations in hydrometeors concentrations are not accurate (Douville 2021). Differences in MP formulations in the schemes have been found to play a vital role on the simulation of the thermodynamic profile that in turn influence the intensity of convection responsible of heavy rainfall \citep{Thomas2021}. \vskip0.2cm

Different MP causes substantial differences in distribution and magnitude of convective precipitation, especially over mountainous areas \citep{Gao2018}. It has been shown that the choice of the MP scheme can result in large differences in the simulation of precipitation types and amount at different altitudes over the TP \citep{Orr2017}. This could be caused, for example, by the latent heat released by phase change processes (both liquid- and ice-phase) which can influence the thermal profile and alter the distribution of moisture in the vertical direction. A study that examined the influences of different cloud MP schemes on summer monsoon precipitation in the central Nepalese Himalayas suggests that the interactions between liquid- and ice-phase hydrometeors are critical to precipitation formation and timing \citep{Orr2017}. Therefore, liquid- and ice-phase MP plays a critical role in clouds and precipitation processes over the TP \citep{Tang2019,Hazra2020}. Nonetheless, our understanding of the internal properties of cloud systems is still scarce and the study of the impacts of ice cloud MP on convective precipitation over the TP remains incomplete. \vskip0.2cm

A research over the Central Himalayas \citep{Karki2018} showed that the selection of an appropriate MP scheme is crucial for the reliable prediction of intense convective rainfall. A recent study that performed ensemble-physics simulations over the TP also suggested that MP schemes are one of the leading sources for modelling uncertainties in terms of precipitation \citep{Prein2022}. While the consequences of using different MP schemes have been pointed out, the processes and the main factors contributing to precipitation differences among the schemes are not well understood. 


\rhead{\textsl{Chapter 1: Introduction}}
\section{Aims and objectives}	
This thesis aims to study how precipitation related processes are represented by different MP schemes and explore the factors leading to uncertainties in precipitation simulations over the TP. Specifically, we focused on two case studies of the Coordinated regional climate downscaling experiment (CORDEX) flagship pilot study Convection-Permitting Third Pole (CPTP) and six MP schemes within WRF. The analysis was divided into three objectives:
\begin{enumerate}
\item Examine the spatial and temporal differences of precipitation using WRF simulations with various MP schemes and testing them against observation and reanalysis datasets.
\item Investigate the simulated atmospheric processes by studying the energy budget, atmospheric instability, moisture transport and hydrometeors distributions. 
\item Identify the key factors leading to uncertainties in precipitation simulations with different MP schemes.
\end{enumerate}



\pagestyle{fancy}
\fancyhead{} % cancella tutti i campi
%\rhead[\textsl{Chapter 2.~~ Data and methods}]{}%\lhead[]{\textsl{1.1.~~CLIMATE CHANGE}} 
\rhead{\textsl{Chapter 1: Introduction}}
\renewcommand{\headrulewidth}{0pt}
\chapter{Data and methods}
\section{Case studies}
We selected two cases of intense precipitation that are part of the CORDEX Flagship Pilot Study CPTP: a summer MCS and a heavy snow event that occurred in early autumn. \vskip 0.2cm

During summer 2008, a long-lasting MCS emerged over the TP. On July 18, a mesoscale Tibetan Plateau Vortex (TPV) formed in the western part of the TP, traveled eastwards, moved off the plateau and then traveled north-eastward to the coast of the Yellow Sea \citep{Feng2014,Curio2019}. When the TPV reached the eastern TP, it triggered a MCS and produced substantial amounts of convective precipitation in the Yangtze river basin \citep{Kukulies2021}. For our project, we focused on the area delimited by 26-35°N and 100-115°E (Fig.\ref{TP2}a) from July 16 2008 0 UTC to July 24 2008 23 UTC. \vskip 0.2cm
In early autumn 2018, a heavy snow event occurred over the TP and it was one of the largest since recording began in 2005. The event took place between October 4 and 8 around the Nam Co station \citep{Dai2020}. On October 3, a gust of cold air passed over the group lakes area, leading the daily minimum temperature of the region to drop by about 4.5℃ within 24 hours. As a result, the daily minimum temperature reached the freezing point about one week ahead of previous years. The amount of snowfall to the east of Nam Co was more than 50 cm while it was only 5 cm in the west. For our research, we focused on the area delimited by 27-35°N and 90-105°E (Fig.\ref{TP2}b)  from October 4 2008 0 UTC to October 9 2018 23 UTC.
\begin{figure}[b!]\centering\includegraphics[width=1\textwidth]{./intro/TP2}\caption{Topography of a) The Sichuan basin b) The Nam Co area. The black contours indicate elevation at 3000 m.}\label{TP2}\end{figure}
\rhead{\textsl{Chapter 2: Data and methods}}
\renewcommand{\headrulewidth}{0pt}

\section{Datasets}
\subsection{Validation datasets}
Virtually all the variables used were tested against ERA5, which is the fifth generation of reanalysis provided by the European Centre for Medium-Range Weather Forecast (ECMWF) for global climate and weather \citep{Hersbach2020}. Reanalysis merges model outputs with observations into a globally complete and consistent dataset using the laws of physics to produce a new best estimate of the state of atmosphere. ERA5 provides hourly estimates for a large number of atmospheric quantities, both on a single level and on pressure levels. The horizontal resolution is 0.25°$\times$0.25° for most variables, while it is 0.5°$\times$0.5° for mean quantitites (Available at https://cds.climate.copernicus.eu). \vskip 0.2cm

For precipitation and temperature at surface (T2m) we also used the Global Land Data Assimilation System (GLDAS), which aims to combine satellite- and ground-based observational data into one product using advanced data assimilation techniques \citep{Rodell2004}.  Observation-based precipitation, radiation products and the best available analyses are employed to force the models. GLDAS provides a number of three-hourly variables products with a horizontal resolution of 0.25°$\times$0.25° (Available at https://ldas.gsfc.nasa.gov/gldas). \vskip 0.2cm

Precipitation was also compared to NASA's Integrated Multi-satellitE Retrievals for GPM (IMERG) and to the Asian Precipitation-Highly Resolved Observational Data Integration Towards Evaluation of Water Resources (APHRODITE) \citep{Huffman2020,Yatagai2012}. IMERG aims to unify and advance precipitation measurements from research and operational microwave sensors for delivering next-generation global precipitation data products. It produces hourly data with a spatial resolution of 0.1°$\times$0.1° globally (Available at https://gpm.nasa.gov/data/imerg). APHRODITE is a set of long-term continental-scale daily products based on rain-gauge data for Asia, with a 0.25°$\times$0.25° spatial resolution (Available at https://climatedataguide.ucar.edu/climate-data). \vskip 0.2cm

It has to be noted that we were able to test our heavy snow system simulation only against ERA5 and IMERG. GLDAS was not used because the dates of the event were not available, while APHRODITE only produces data up to 2015.


\subsection{Weather Research and Forecast model}
The Advanced Research (AR) Weather Research and Forecasting (WRF) model version 4.2 was used to produce the simulations \citep{Skamarock2019,POwers2017}. WRF is a state-of-the-art mesoscale numerical weather prediction system that is widely used for modelling across weather and climate time scales. WRF uses fully-compressible, Eulerian non-hydrostatic equations that conserve dry air mass and scalar mass. It uses a staggered Arakawa C-grid with a terrain-following, mass-based, hybrid sigma-pressure vertical coordinate system with a vertically stretched grid \citep{Prein2022}. Our simulations are performed using six different MP schemes and run at 4 km convection-permitting scale. The MP schemes involved are: Morrison 2-moment (Here on Morrison), Thompson, CAM 5.1 (Here on CAM), SBU-YLin (Here on Ylin), WDM6 and WDM7. We were not able to complete our simulations of the heavy snow system using CAM MP scheme. Therefore, for the second case study we have five simulations in total instead of six.

\subsection{Microphysics schemes}
There are two distinct strategies to simulate cloud MP: the explicit bin resolving method and the bulk method. Bin resolving MP models explicitly calculate the particle size distribution and therefore provide more accurate simulations than bulk models. However, the computational demand of bin MP schemes limits their application in climate models. In alternative, bulk MP schemes provide a simpler and computationally efficient approach that predicts several drop size distribution (DSD) moments rather than the DSD itselfs. The bulk method can be classified into two categories: single-moment approach and multiple-moment approach. In the first case, the MP scheme predicts only the mixing ratios of the hydrometeors by representing the size for each class with a function, for instance a gamma or an exponential distribution. Instead, in the second case, the MP scheme predicts not only the mixing ratio of the hydrometeors but also another quantity, such as their number concentrations \citep{Lim2010}. WRF MP schemes all use the double-moment bulk method to simulate multiple hydrometeors species mixing ratios and number concentrations, while ERA5 is based on a single-moment bulk scheme to predict the hydrometeors mixing ratios.

\subsubsection{Thompson}
%gamma capital is factorial function: gamma(n) = (n -1)!
\begin{figure}[t!]\centering\includegraphics[width=0.7\textwidth]{./intro/tho}\caption{Flowchart diagram summarising all the MP processes of the cloud water, rainwater, cloud ice, snow, and graupel in Thompson scheme \citep{Bao2016}.}\label{tho2}\end{figure}
Thompson MP scheme provides a description of water vapour and five hydrometeors species: cloud liquid and ice water, rain, snow and graupel \citep{Thompson2008}. Each species, except snow, follows a generalised gamma distribution (Eq.\ref{Dis}), where $\Gamma$ is Euler gamma function, N$_{t}$ is the total number of particles, D is the particle diameter, $\lambda$ is the distribution's slope and $\mu$ is the shape parameter. When $\mu$ is zero, the distribution becomes a classic exponential. Unique to this MP scheme, $\mu = 0$ is not directly associated with any water species but depends also on the formation process. This means that, for example, the rain size distribution significantly varies depending on whether the rain originates from melted ice versus rain produced by the collision–coalescence (warm rain) process. 
\begin{equation} \label{Dis} N(D) = \frac{N_{t}}{\Gamma(\mu +1)} \lambda^{\mu + 1} D^{\mu} e^{-\lambda D}\end{equation}
Water and ice species use the standard power-law form to describe their mass and terminal velocity as a function of diameter (Eq.\ref{MD}), where $a, b$ are mass constants, $\rho$ is the moist air density, $\rho_{0}$ is the reference air density and $\alpha, \beta, f$ are velocity constants. Only cloud water is assumed not to sediment and the two liquid species use the spherical assumption where $\rho_{w}$= 1000 kg m$^{-3}$ is the density of water. Snow size distribution depends on both ice water content and temperature, and is represented as a sum of exponential and gamma distributions. Furthermore, snow assumes a nonspherical shape with a bulk density that varies inversely with diameter. 
\begin{equation}\label{MD} a)~m(D) = a D^{b}~~~~~~~~~b)~v(D) = \left(\frac{\rho_{0}}{\rho}\right)^{1/2} \alpha D^{\beta} e^{-fD}\end{equation}
Each species mass mixing ratio or number concentration follows the same governing conservation equation Eq.\ref{tho1}, where $\Phi$ is mass mixing ratio or number concentration, $t$ is time, $\rho$ is the air density, $U$ is the wind vector, $z$ is height, $V_{\Phi}$ is the fall speed, $\delta_{\phi}$ is the subgrid-scale mixing operator and $S_{\Phi}$ represent the various MP processes rates \citep{Thompson2014}.
\begin{equation}\label{tho1} \frac{\partial \Phi}{\partial t}  = -\frac{1}{\rho}\nabla \cdot (\rho U \Phi) - \frac{1}{\rho}\frac{\partial(\rho V_{\Phi}\Phi)}{\partial z} + \delta \Phi + S_{\Phi} \end{equation}
This scheme includes the description of main microphysical processes involved in precipitation production, including warm rain processes, ice growth mechanisms and hydrometeors sedimentation. This means that the following processes are simulated: condensation and evaporation, autoconversion, ice nucleation, cloud ice conversion to snow, deposition and sublimation of ice species, hydrometeors collision and collection, conversion of rimed snow to graupel, melting of ice species and sedimentation of hydrometeors \citep{Thompson2008}. A flowchart diagram summarising all MP processes of cloud water, rainwater, cloud ice, snow and graupel is presented in Fig.\ref{tho2} \citep{Bao2016}.

\subsubsection{Morrison}
\begin{figure}[t!]\centering\includegraphics[width=0.7\textwidth]{./intro/morr}\caption{Flowchart diagram of MP processes of the water species. In red the latest improvements to the Morrison scheme \citep{Gettelman2019}.}\label{mor1}\end{figure}
Morrison MP scheme simulates the mass mixing ratios and the number concentrations of water vapour and five hydrometeors species, namely cloud liquid and ice water, rain, snow and graupel. \citep{Morrison2009}. The size distributions are represented by gamma functions, defined in Eq.\ref{DMD}, where N$_{0}$, $\lambda$ and $\mu$ are the intercept, slope and shape parameters, and D is the particle diameter. The parameters N$_{0}$ and $\lambda$ are derived from the the predicted number concentration N and mixing ratio $q$ and defined in Eq.\ref{DMD}, where $c, d$ are constants given by the law mass-diameter (Eq.\ref{MD} where a is c and b is d). All particles are assumed spheric. For the precipitable species (i.e. rain, snow and graupel) and cloud ice, $\mu$ = 0, meaning that these species follow an exponential distribution.
\begin{equation}\label{DMD} a)~N(D) = N_{0} D^{\mu} e^{-\lambda D}~~~~~~b)~\lambda = \left[\frac{c N \Gamma(\mu + d + 1)}{q \Gamma(\mu + 1)}\right]^{1/d}~~~~~~c)~N_{0} = \frac{N \lambda^{\mu + 1}}{\Gamma(\mu + 1)}\end{equation}
The temporal evolution of mixing ratios ($q$) and number concentration follow the same equation Eq.\ref{mor}, where $v$ is the wind vector, $V_q$ is the terminal fall speed, $\nabla_D$ is the turbolent diffusion operator to parametrise turbolent mixing and $S$ represent the rate of all the other MP processes \citep{Morrison2005}.
\begin{equation}\label{mor} \frac{\partial q}{\partial t}  = -\nabla \cdot (vq) + \frac{\partial}{\partial z}(V_q) + \nabla_D q + S \end{equation}
In general, the change in one species concentration due to physical transformations (for instance melting and sublimation) is computed by assuming that the change in number concentration equals the variation in mixing ratios \citep{Morrison2009}. Fig.\ref{mor1} shows the relationship between the water species and the MP processes between them \citep{Gettelman2019}.

\subsubsection{CAM}
\begin{figure}[t!]\centering\includegraphics[width=0.7\textwidth]{./intro/camm}\caption{Flowchart diagram of MP processes of the water species in CAM scheme \citep{Salzmann2010}.}\label{camm}\end{figure}
CAM MP scheme is based loosely on Morrison \citep{Neale2012}. It predicts the number concentrations and mixing ratios of water vapour and four hydrometeors: cloud liquid and ice water, rain and snow. Graupel is also included in the scheme but in our project its mixing ratio was not simulated. Similarly to Morrison, the size distributions are represented by gamma functions (Eq.\ref{cam}), where D is the diameter, N$_{0}$ is the intercept parameter, $\lambda$ is the slope parameter and $\mu = 1/\eta^2 - 1$ is the spectra shape parameter with $\eta$ being the relative radius dispersion of the size distribution. For cloud ice $\mu = 0$.
\begin{equation}\label{cam} a)~N(D) = N_{0} D^{\mu} e^{-\lambda D}~~~~~~b)~\lambda = \left[\frac{\pi \rho N \Gamma(\mu + 4)}{6 q \Gamma(\mu + 1)}\right]^{1/3}~~~~~~c)~N_{0} = \frac{N \lambda^{\mu + 1}}{\Gamma(\mu + 1)}\end{equation}
The time evolution of mixing ratios ($q$) and number concentrations follow Eq.\ref{cam1}, where $z$ is height, $V$ is terminal fall speed and $S$ accounts for the sink/source terms \citep{Morrison2008}.
\begin{equation}\label{cam1} \frac{\partial q}{\partial t} = S + \frac{1}{\rho}\frac{\partial}{\partial z} (\rho V q) \end{equation}
The mechanism considered in the scheme include activation of cloud condensation nuclei or deposition/condensation-freezing nucleation on ice nuclei to form droplets or cloud ice; ice multiplication via rime-splintering on snow; condensation/deposition, evaporation/sublimation, autoconversion of cloud droplets and ice to form rain and snow, accretion of cloud droplets and ice by rain, accretion of cloud droplets and ice by snow, heterogeneous freezing of droplets to form ice, homogeneous freezing of cloud droplets, melting, ice multiplication, sedimentation, and convective detrainment \citep{Neale2012}. Fig.\ref{camm} represents a summary of the MP processes included \citep{Salzmann2010}.

\subsubsection{Ylin}
\begin{figure}[t!]\centering\includegraphics[width=0.9\textwidth]{./intro/ylin}\caption{Flowchart of the MP processes in Ylin scheme for (a) Mass mixing ratio (b) Number mixing ratio \citep{Zhao2021}.}\label{ylin2}\end{figure}
Ylin MP scheme includes water vapour and four hydrometeors: cloud liquid and ice water, rain and precipitating ice \citep{Lin2011}. Snow and graupel share the same category as well as the same processes (deposition/sublimation and collision with other hydrometeors). This approach leads to a reduction of the number of microphysical processes that need to be computed and thus to a limitation of computational time. Similarly to previous schemes, also Ylin uses a generalised gamma function to describe the size distribution of the water species (Eq.\ref{Ylin}), where $N_{0c}$ is the intercept, $\mu$ is the shape parameter and $\lambda$ is the slope. The total number concentration of cloud droplets generally depends also on the ambient aerosol distribution and properties.\\
\begin{equation}\label{Ylin} N_c = N_{0c} D^{\mu} e^{-\lambda D}\end{equation}
The governing equation of the mass and number mixing ratios ($q$) for each species is given by Eq.\ref{ylin1}, where $u$ is the wind vector, $u \cdot \nabla q$ represents advection, the first term at the right side refers to sedimentation processes and S represents the rate of all the other MP processes \citep{Zhao2021}.
\begin{equation}\label{ylin1} \frac{\partial q}{\partial t} + u \cdot \nabla q = \frac{1}{\rho} \frac{\partial (\rho q V_q)}{\partial z} + S \end{equation}
The collision and coalescence of cloud droplets to form raindrops is described using a simple autoconversion process, using the cloud liquid water content. The distinction between cloud and precipitating ice is not as physically based as the division between liquid cloud droplets and rain, which grow by condensation and collision-coalescence, respectively. Therefore, a maximum size of 100 $\mu$m is applied for cloud ice to snow conversion in this scheme. Moreover, to simplify the description of ice variables, a monodisperse distribution is applied \citep{Lin2011}. Fig.\ref{ylin2} summarises all MP processes in this scheme \citep{Zhao2021}.

\subsubsection{WDM6}
\begin{figure}[t!]\centering\includegraphics[width=1\textwidth]{./intro/wdm6flow}\caption{Flowchart of the microphysics processes for the prediction of (a) the mixing ratios and (b) the number concentrations in the WDM6 scheme. The terms with red (blue) colors are activated when the temperature is above (below) 0°C, whereas the terms with black color are in the entire regime of temperature \citep{Lim2010}.}\label{wdm6flow}\end{figure}
WDM6 MP scheme predicts the mixing ratios and number concentrations of water vapour and five hydrometeors (cloud liquid and ice water, snow, rain and graupel) together with a prognostic variable of CCN number concentration \citep{Lim2010}. The size distribution is assumed to follow Eq.\ref{wdm6}, where $\lambda$ is the corresponding slope parameter, $\nu, \alpha$ are two dispersion parameters, $N_{0}$ is the total number concentration and $D$ is the diameter of the species. The dispersion parameters for the size distribution of rain are chosen as $\nu = 2$ and $\alpha = 1$ , while for the cloud water size distribution $\nu = 1$ and $\alpha = 3$.
\begin{equation}\label{wdm6} a)~N(D) = N_{0} \frac{\alpha}{\Gamma(\nu)} \lambda^{\alpha \nu} D^{\alpha \nu - 1} e^{-(\lambda D)^\alpha}~~~~~~~b)~\lambda = \left[\frac{\pi}{6} \rho_{w} \frac{\Gamma(\nu + 3/\alpha)}{\Gamma(\nu)} \frac{N_{0}}{\rho q}\right]^{1/3}\end{equation}
The hydrometeors mixing ratios ($q$) follow Eq.\ref{wdm6}, where $V$ is the wind vector, $V \cdot \nabla q$ represents advection, the first term at the right side refers to sedimentation processes and S represents the rate of all the other MP processes \citep{Hong2006TheWS}.
\begin{equation}\label{ylin1} \frac{\partial q}{\partial t} + V \cdot \nabla q = - \frac{q}{\rho} \frac{\partial (\rho V_q)}{\partial z} + S \end{equation}
Fig.\ref{wdm6flow} shows a flowchart of the microphysics processes for the prediction of the mixing ratios and the number concentrations in the WDM6 scheme. One of the distinct features of this MP scheme is that the activated CCN number concentration is predicted and formulated by the drop activation process based on the relationship between the activated number of CCN and supersaturation. This adds a level of complexity to the traditional bulk MP schemes through the explicit CCN cloud-drop concentration feedbacks \citep{Lim2010}. %It has been shown in previous studies \citep{Lim2010} that comparing WDM6 and Morrison simulations of a precipitation event led to substantial differences. The determination of what caused these discrepancies is highly complex, but the absence of enhanced melting processes of snow and graupel in Morrison could be a relevant contribution.

\subsubsection{WDM7}
\begin{figure}[t!]\centering\includegraphics[width=0.6\textwidth]{./intro/wdm7}\caption{Flowchart of the MP processes in WDM7 scheme \citep{Bae2018}.}\label{wdm7}\end{figure}
WDM7 is developed by introducing the hail hydrometeor as an additional prognostic water substance within the WDM6 scheme \citep{Bae2018}. WDM7 scheme with hail tends to enhance the accretion rate of ice particles due to the faster sedimentation of hail than that of graupel in the WDM6 scheme. The amount of hail leads to the reduction of graupel. Less intense accretion of graupel by snow at higher altitudes maintains the snow aloft and increases the snow content at mid-level. The reduced sum of graupel and hail at the melting level causes a decrease in the rain mixing ratio in WDM7, which is compensated by falling hail. This scheme tends to enhance convective activities in the leading edge of the convective region, whereas the precipitation intensity in the stratiform region decreases. This is due to the fact that the addition of hail plays a role in suppressing light precipitation and increasing heavy precipitation activities. Fig.\ref{wdm7} summarises all MP processes described in this scheme.\\
The hail size distribution is assumed to follow the exponential form Eq.\ref{wdm7}, where $n_H(D_H)dD_H$ is the number of hail particles, $n_{0H}$ is the intercept parameter and $\lambda_H$ is the slope of the distribution \citep{Bae2018}.
\begin{equation}\label{wdm7} n_H(D_H)dD_H = n_{0H} exp(-\lambda_H D_H)dD_H \end{equation}

\subsubsection{ERA5}
ERA5 MP scheme is based on Tiedke MP scheme, which has undergone significant improvements since its original release \citep{Forbes2011}. Some of the main developments include ice sedimentation and autoconversion to snow, subgrid precipitation coverage and evaporation, numerical treatment of the cloud condensate and cloud fraction equations and representation of ice supersaturation in cloudless air. The latest developments include also precipitation advection and mixed-phase cloud processes. Currently, this scheme predicts the mixing ratios of water vapour and four hydrometeors: cloud liquid and ice water, rain and snow. The philosophy of the original Tiedtke scheme is mantained with regards to prognostic cloud fraction and sources/sinks of all cloud variables due to the major generation and destruction mechanisms. Fig.\ref{eraMP} summarises the MP processes taken into consideration in this scheme.
\begin{figure}[t!]\centering\includegraphics[width=0.5\textwidth]{./intro/era5MP}\caption {Flow chart of the Tiedtke scheme with six moisture related prognostic variables in ERA5 \citep{Forbes2011}.}\label{eraMP}\end{figure}
The equation governing each prognostic variable within this MP scheme is Eq.\ref{Tied}, where $q_x$ is the specific water content for category $x$, $S_x$ the net source or sink of $q_x$ through microphysical processes, $z$ is height and the last term to the right represents the sedimentation of $q_x$ with fall speed $V_x$.
\begin{equation}\label{Tied} \frac{\partial q_x}{\partial t} = S_x + \frac{1}{\rho}\frac{\partial}{\partial z} (\rho V_x q_x) \end{equation}


\section{Methods} 
\subsection{Pre-processing}
WRF simulations were processed and analyzed using the high-level programming language Python, the command line software CDO (Climate Data Operators, available at https://code.mpimet.mpg.de/projects/cdo) and  by bash scripting. WRF outputs contained a large variety of parameters of different nature, therefore we extracted the variables we were interested in and saved them into new NetCDF files. \vskip 0.2cm
Two-dimensional (2D) variables were then cut for the specific temporal and spatial domain and were ready to use. In case we needed to compute the difference between WRF runs and other datasets, such as for precipitation, we regridded the data to the grid of lower resolution by finding the nearest points. Moreover, for some variables, namely temperature at surface (T2m) and outgoing longwave radiation (OLR), we also analysed the time evolution after having removed the diurnal cycle in order to evaluate how the MP schemes reproduced hourly deviations. The diurnal cycle was computed by averaging all WRF runs and ERA5 together, and then it was substracted to each dataset. \vskip 0.2cm 
Three-dimensional (3D) variables, because of the additional vertical dimension, needed to be regridded to ERA5 latitude and longitude grid from the start. In this way, their horizontal resolution was lowered but the vertical resolution was kept unaltered, and they became computationally easier to analyse. Depending on the variable, two types of vertical grids were used: non-staggered and staggered. In the first case the vertical levels were 49 while in the second case they were 50. To be able to compare the data, we interpolated the staggered levels to produce 49 vertical levels and convert them into the non-staggered grid. Moreover, since WRF uses terrain-following vertical coordinates, we had to compute the equivalent pressure levels using Eq.\ref{PPB}, where $L$, $PresB$ and $Pres$ are the new pressure levels, the base state pressure and perturbation pressure at half-sigma levels respectively, and divided by 100 to have hPa as unit of measure. 
\begin{table}[!b]\caption {Summary of variables used. Variables that needed to be alterated or computed before the analysis are indicated with an asterisk.}\centering\includegraphics[width=1\textwidth]{./resultsMCS/variables}\label{var}\end{table}
Then, we interpolated $L$ to obtain the same pressure levels of ERA5, which resulted in 29 vertical levels ranging from 975 hPa to 30 hPa for the MCS case and 25 vertical levels ranging from 875 hPa to 30 hPa for the heavy snow system case.
\begin{equation}\label{PPB} L = (PresB + Pres)/100 \end{equation}

\subsection{Variables definition}
All variables used are listed in Table \ref{var}. Accumulated precipitation, liquid water path (LWP) and ice water path (IWP) were computed starting from other variables for each dataset. Accumulated precipitation is the sum of hourly (or three-hourly in case of APHRODITE) precipitation rate. LWP and IWP were computed using Eq.\ref{eqLWP} \citep{Chen2015}, where $g$ is the gravity acceleration constant equal to $\sim$9.81 $m/s^2$; $Q_{sum}$ is the sum of liquid hydrometeors for LWP (namely Qcloud and Qrain) and the sum of solid hydrometeors for IWP (namely Qice, Qsnow, Qgraupel and Qhail); $dp$ is the difference in pressure between atmospheric levels.
\begin{equation}\label{eqLWP} LWP = \frac{1}{g} \int_{L_{min}}^{L_{max}} Q_{sum} \, dp \end{equation}
Horizontal winds were computed using the eastward (U) and norhtward (V) components of wind through $calc.wind\_speed$, a specific Python function from $metpy$ package (Available at https://pypi.org/project/MetPy). W and TH were produced by WRF but not by ERA5, so they were computed using specific functions from $metpy$ package: $calc.vertical\_velocity$ and $calc.potential\_temperature$ respectively. 




\chapter{Results}
\rhead{\textsl{Chapter 3: Results}}
\renewcommand{\headrulewidth}{0pt}
The analysis is divided into two main sections for each study case. Firstly, we consider the differences in time evolution and spatial patterns of 2D variables, starting from precipitation, temperature at surface (T2m) and outgoing longwave radiation (OLR). In this way, we illustrate the variations in the simulations caused by MP schemes and establish which variables exhibit the biggest discrepancies. We also identify relevant relationships with other 2D variables that affect the precipitation outcome and highlight how they are differently portrayed in the simulations. The second step consists in the analysis of the vertical profiles of 3D variables, such as vertical velocity (W), mixing ratios of the hydrometeors and more.

\section{MCS case} %20 if also snow is done, 30 if snow is not done = results + discussion
\subsection{Precipitation, outgoing longwave radiation and temperature} %tot 10
The primary focus of this section will be on the evaluation of the six MP schemes representation of the MCS and on understanding how variables participating in the precipitation processes can give indication of the accuracy of the simulation. Alongside the temporal and spatial analysis of precipitation, we will study OLR and T2m as they give an indication of the presence of convective clouds and of the surface heating respectively.

\subsubsection{Temporal evolution} %2
The time series of precipitation rate is presented in Fig.\ref{PPOLRT}a. As we can observe, WRF simulations display major differences in intensity and timing based on the MP scheme used and also compared to validation datasets. In fact, ERA5, IMERG, GLDAS and APHRODITE are in good agreement when reproducing the main precipitation peak both in terms of intensity (16-20 mm/day) and timing (July 21), and show also a minor peak on July 17. In contrast, in WRF simulations the major precipitation peak is one or two days in advance and the minor peak does not appear. It is clear that the choice of the MP scheme heavily influences the precipitation output, as some schemes - Thompson, Morrison and Ylin - have a narrow and intense peak similar to validation datasets, while the rest - CAM, WDM7 and WDM6 - display a much broader and weaker peak. Even though the simulation of T2m is much more coherent (Fig.\ref{PPOLRT}b), some of the previous patterns still apply. In fact, the minimum T2m in WRF simulations is still earlier than that in the other datasets. Compared to GLDAS, ERA5 overestimates the T2m of about 1-2 °C but the temporal evolution is very similar. For every dataset, the minimum in T2m is roughly one day after the correspondent peak in precipitation. In Fig.\ref{PPOLRT}c, the OLR time series is illustrated. Clear differences arise considering the magnitude of the minimum: WDM7, WDM6 and CAM have broad and weak minima around 240 W m$^{-2}$, whereas Thompson, Morrison and Ylin exhibit much deeper sinks ranging from 155 to 200 W m$^{-2}$. In particular, Ylin has the lowest OLR throughout the whole event. Moreover, the correspondence in time between minima in OLR and peaks in precipitation rate confirms that intense precipitation was caused by convective cloud systems.
\begin{figure}[t!]\centering\includegraphics[width=1\textwidth]{./resultsMCS/precipolrt}\caption{Time series averaged over spatial domain (26-35°N and 100-115°E) of a) Daily mean precipitation rate, b) Daily mean T2m, c) Daily mean OLR and d) Hourly accumulated precipitation. The dashed lines identify validation datasets, while the solid lines are used for WRF runs. } \label{PPOLRT} \end{figure}
The differences in precipitation rate are reflected also in the accumulated total precipitation (Fig.\ref{PPOLRT}d). ERA5, IMERG, GLDAS and half of the WRF simulations (Thompson, Morrison and CAM) show a similar value at the end of the event, which is around 70-80 mm; while for the rest of the WRF simulations (WDM7, WDM6 and Ylin) and APHRODITE it is substantially lower, with values around 50-60 mm. \vskip 0.2cm
\begin{figure}[t!]\centering\includegraphics[width=1\textwidth]{./resultsMCS/corrs}\caption{Correlograms of hourly temporal evolution averaged over spatial domain (26-35°N and 100-115°E) of a) precipitation rate, b) T2m, c) OLR and d) Accumulated precipitation.} \label{CORR} \end{figure}

The correlograms presented in Fig.\ref{CORR} provide additional insight on how the temporal evolution is represented in WRF simulations and validation datasets. The precipitation correlogram gives a clear indication that the MP scheme that reproduces the time series most similarly to ERA5 and IMERG is Ylin with correlation coefficients of 0.58-0.68, followed by Thompson and Morrison respectively with 0.45-0.6 and 0.44-0.56. The differences displayed in the time series are reflected also in the correlation coefficients, that among the schemes vary from 0.59 to 0.96, with Thompson and Morrison having the highest correlation between themselves. The correlation coefficients for T2m are definitely higher, with the lowest value being 0.9 relative to CAM and ERA5. Among WRF simulations the coefficients are always above 0.97, meaning that the choice of MP does not cause major differences in the simulation of T2m. Considering the correlogram of OLR, the schemes in better agreement with ERA5 are WDM6 and WDM7 with values of 0.71 and 0.69 respectively, while the worst is Ylin with 0.56. Moreover, we find relevant variations also among the schemes correlation coefficients, with values ranging from 0.56 to 0.95. It is interesting to note that the performance of MP schemes differs when removing the diurnal cycles of T2m and OLR (Fig.\ref{decy}). In fact, the choice of MP scheme leads to variations in simulated temperature and OLR of about 1-2°C and 100 W m$^{-2}$ respectively. Moreover, the negative correlations values are mainly due to the temporal shift of about one day between ERA5 and WRF runs when reproducing features related to precipitation peak. Lastly, the correlogram of accumulated precipitation (Fig.\ref{CORR}d) shows that this variable is simulated coherently among all datasets, as the lowest correlation coefficient is 0.96 relative to WDM7 and ERA5. \\

 
\subsubsection{Spatial patterns} %3
\begin{figure}[t!]\centering\includegraphics[width=0.97\textwidth]{./resultsMCS/POLR_maps}\caption{Spatial distribution averaged over July 16-24 of a) precipitation, b) OLR.} \label{POLRmaps} \end{figure}
Since the relationship between precipitation and OLR is more indicative for precipitation formation processes, here on we will focus exclusively on precipitation and OLR, leaving aside T2m. Nonetheless it is important to remember that, despite being relatively small (mostly below 1°C), the differences in T2m could still contribute to the differences in simulated precipitation by WRF. Moreover, we decided to use as validation datasets only IMERG and ERA5, since they have the highest spatial and temporal resolution. \vskip 0.2cm
In Fig.\ref{POLRmaps} the spatial distributions of precipitation and OLR are shown. The spatial pattern of precipitation is vastly different depending on the MP scheme. The intensity of precipitation observed in the time series finds confirmation in the maps. In fact, Thompson and Morrison display a clear precipitation peak around 29°N and 110°E, WDM7 and WDM6 show a less intense and more spread out peak, while Ylin and CAM are somewhere in between. OLR maps show a similar spatial pattern, in which the lowest values correspond to precipitation maxima. As for the time evolution, also the location of intense precipitation is connected to the position of convective clouds. However, it has to be noted that the sinks in OLR are not proportional to peaks in precipitation for all the schemes: while it is true that Thompson and Morrison exhibit lower values of OLR than WDM7 and WDM6, Ylin values are disproportionate to the intensity of the correspondent precipitation peak. In addition, topography clearly plays a role both in precipitation and OLR as we can identify from their spatial patterns where the Sichuan basin is located. \vskip 0.2cm
\begin{figure}[t!]\centering\includegraphics[width=0.97\textwidth]{./resultsMCS/POLRmapsobs}\caption{Spatial distribution averaged over July 16-24 of a) precipitation according to ERA5 and IMERG, b) OLR according to ERA5.} \label{POLRmapsERA} \end{figure}
From the comparison of WRF patterns (Fig.\ref{POLRmaps}) with IMERG and ERA5 (Fig.\ref{POLRmapsERA}), differences in distribution and intensity are observable. It is also evident that IMERG and ERA5 capture the precipitation peak more north than in WRF simulations and with a different spatial pattern. For OLR, ERA5 shows similar distribution and values to Thompson and Morrison, being substantially higher than Ylin and lower than WDM6. This similarity translates to similar precipitation patterns, which in ERA5, Thompson and Morrison exhibit good agreement. %Another relevant information to be mentioned is the obvious difference in spatial resolution of the datasets, which is higher for WRF runs and IMERG, and lower for ERA5. \vskip 0.2cm
\begin{figure}[t!]\centering\includegraphics[width=1\textwidth]{./resultsMCS/diffrimergP}\caption{Differences in precipitation spatial distribution (averaged over July 16-24) using IMERG as reference.} \label{Pdiff} \end{figure}
The differences in precipitation spatial patterns can be seen more easily in Fig.\ref{Pdiff}, where IMERG is used as reference. Here, we can see that all the other datasets underestimate precipitation intensity towards the northeast side of the domain. ERA5 appears to be the closest by intensity to IMERG. However, particularly for Thompson, Morrison and Ylin, immediately south of the clear negative difference there is an equivalently clear positive difference with a similar shape to the precipitation peak of IMERG and ERA5. This indicates that the schemes could be representing the spatial distribution of precipitation slightly shifted southward. Therefore, the spatial pattern itself would not be so different but only shifted in location. However, for the rest of the schemes this reasoning can be applied only partially because they display a different pattern of positive precipitation difference. \vskip 0.2cm

It is thus clear that MP schemes have a relevant impact on the temporal and spatial simulations of precipitation and OLR. Nonetheless, the accuracy of the simulation is also dependent on additional factors, such as the extent of the spatial domain. In fact, simulations over the whole TP region generally perform better at capturing precipitation, temperature and OLR and the variations between runs are smaller. On the contrary, when the domain is restricted to the area of interest for the MCS, the simulations display major differences. This means that, even though the model is run at convection permitting scale, it still performs better over larger domains and struggles to reproduce local phenomena.

\subsection{Analysis of other 2D variables}%5
In the following paragraphs, we will expand our analysis to other variables that participate in the precipitation process, such as convective available potential energy (CAPE), planetary boundary layer height (PBLH) and moisture flux at the surface. In fact, CAPE is a measure of the amount of energy available in the atmosphere and thus is directly related to the development of intense convective activity; high values of PBLH indicate more potential for convective activity; while moisture flux at surface is an indicator of moisture content in the lower atmosphere that can significantly contribute to atmospheric instability. 
% CHECK them - cape is a measure of the available energy for the storm? - pblh is important bc the height influences the space for convection but at the same time huiwen said that the rain brings everything down, so I guess when its precipitation peak is right that pblh is at its minimum - moisture flux cause obv gives an indication of the moisture that's going up.

\subsubsection{Convective available potential energy}
\begin{figure}[t!]\centering\includegraphics[width=1\textwidth]{./resultsMCS/cape}\caption{CAPE spatial distribution averaged over time (July 16-24) and daily time evolution averaged over spatial domain (26-35°N and 100-115°E).} \label{cape} \end{figure}
In Fig.\ref{cape} the spatial pattern and temporal evolution of CAPE are presented. Both for WRF simulations and ERA5, the variable is at its highest a couple of days before the peak of precipitation, at its lowest during the peak and it starts growing again after that. WRF simulations show CAPE values peaking around 200-300 J kg$^{-1}$, which is two to three times lower than the maximum of ERA5 ($\sim$700 J kg$^{-1}$). Since reanalysis values are much higher, the differences between the schemes do not seem so large but they must not be ignored. In fact, the variations between highest and lowest values are of about 100 J kg$^{-1}$, which represents 30-50\% of the absolute values. Schemes that report the highest CAPE are also the ones that simulate more intense precipitation peaks, namely Thompson and Morrison, while schemes with relatively low CAPE exhibit also relatively weak precipitation peaks. \vskip 0.2cm
The spatial pattern is coherent in all WRF runs and ERA5. It is clearly influenced by topography, as the highest values are found in lower elevation regions. Another interesting insight comes from the comparison with precipitation patterns: in fact, the location of precipitation peaks does not correspond to the location of maximum CAPE values. 

\subsubsection{Planetary boundary layer height}
\begin{figure}[t!]\centering\includegraphics[width=1\textwidth]{./resultsMCS/pblh}\caption{PBLH spatial distribution averaged over time (July 16-24) and daily time evolution averaged over spatial domain (26-35°N and 100-115°E).} \label{pblh} \end{figure}
The spatial distribution and time series of PBLH are presented in Fig.\ref{pblh}. The time series illustrates different behaviours of the variable represented in ERA5 versus WRF simulations. In fact, in WRF runs the height reaches its maximum two days before the precipitation peak while it is at its minimum one day after the precipitation maximum. Differences in intensity between MP schemes go up to 100 m, corresponding to 20-30\% of the absolute values. Even though the overall behaviour of PBLH is similar in all WRF simulations, there is no direct relationship between PBLH and precipitation intensity, since schemes with higher PBLH do not simulate more intense precipitation. In contrast to WRF, PBLH in ERA5 rises until it reaches a value of about 500 m and then stays constant for about two days after which it starts lowering again, displaying a completely different temporal evolution from WRF runs. This substantial difference could be due to variations in the definitions of PBLH in WRF versus ERA5. \vskip 0.2cm
The spatial patterns are clearly influenced by topography and are coherent between WRF simulations but exhibit some variations if compared to ERA5. In fact, in the first case PBLH displays the highest values in the northern part of the domain, while in the second case the biggest values are found in the southeast and northwest corners of the domain. %needs to be checked!!!

\subsubsection{Moisture flux at surface}
\begin{figure}[t!]\centering\includegraphics[width=1\textwidth]{./resultsMCS/moisture}\caption{Moisture flux at surface spatial distribution averaged over time (July 16-24) and daily time evolution averaged over spatial domain (26-35°N and 100-115°E).} \label{qflux} \end{figure}
Fig.\ref{qflux} displays the spatial pattern and time evolution of moisture flux at surface. Values are in the order of magnitude of 10$^{-5}$ kg m$^{-2}$s$^{-1}$ and are always positive, indicating that the flux is directed upward. A clear sink in the flux is caused by the peak in precipitation in all datasets, which obviously leads to a decrease in moisture moving upward in the atmosphere. Comparing ERA5 and WRF simulations, the magnitude of the flux is generally in agreement, except for Ylin that stands out for having the lowest minimum. Nonetheless, there is no evident direct relationship between intensity of moisture flux and intensity of precipitation. This could mean that their relationship is more complex and could involve other variables. \vskip 0.2cm
Spatially, the pattern of the moisture flux is heavily influenced by the topography of the region. Lower elevations are characterised by higher values of moisture flux, while higher elevations show weaker upward flux. WRF runs show different spatial patterns both when compared between themselves and to ERA5. Moreover, ERA5 spatial pattern of precipitation and moisture flux display the maximum and minimum values respectively in the same area, indicating that when precipitation occurs the upward moisture flux diminishes. Nonetheless, this relationship is not as evident in WRF simulations, in which precipitation and moisture flux show major variations in their spatial distributions.

\subsection{Vertical profile of atmosphere}%tot 10
The main focus of this section will be on the analysis of the vertical profile of multiple atmospheric variables relevant for precipitation formation and development. The aim is to identify which variables vary the most depending on the MP scheme and the consequences on the simulation of precipitation. For this reason, the vertical velocity (W), potential temperature (TH), specific humidity - also referred to as water vapor mixing ratio - (Q), hydrometeors mixing ratios, liquid/ice water paths (LWP/IWP), cloud fraction and horizontal winds are considered. W is a representation of convection since it indicates the intensity of the updraft or downdraft. TH is a useful indicator of the stability of atmosphere. Q refers to the amount of water vapour in the atmosphere and therefore it is fundamental to understand cloud formation. Hydrometeors, LWP and IWP give information on the cloud phase and the precipitation formation processes as well as on radiative feedbacks. Cloud fraction measures the portion of atmosphere covered by clouds. Lastly, horizontal winds can play a relevant role in the development of convective systems and influence the location of precipitation. 

\subsubsection{Vertical velocity, potential temperature and specific humidity}%2.5
\begin{figure}[t!]\centering\includegraphics[width=1\textwidth]{./resultsMCS/WTHQ}\caption{a) Vertical profile of W, TH and Q (from left to right) averaged over July 16-24, latitudes 26-35°N and longitudes 100-115°E. b) Vertical profile time evolution of ERA5 W, TH and Q (averaged over latitudes 26-35°N and longitudes 100-115°E).} \label{WTHQ} \end{figure}
The vertical profiles of W, TH and Q are illustrated in Fig.\ref{WTHQ}a. The vertical profile of W demonstrates that there are evident variations in its representation but all values are positive, indicating an overall upward convection. ERA5 shows a single peak of about 0.012 m/s at 200 hPa, whereas MP schemes mostly simulate multiple and weaker peaks. Thompson is the most similar to ERA5, with a single peak of about 0.011 m/s at 200 hPa, whereas the most different are WDM7 and WDM6, since they show three distinct and less intense maxima at 100 hPa, 400 hPa and 700 hPa of 0.003 m/s, 0.005 m/s and 0.005 m/s respectively. The rest of the schemes display an intermediate trend. The entire vertical profile of TH is extremely coherent among all datasets, with values ranging from 350 K at 200 hPa to 300 K at 975 hPa. Lastly, all datasets agree on the simulation of Q until 800 hPa, where ERA5 and WRF simulations diverge. In fact, near the Earth's surface, ERA5 exhibits a value of $\sim$0.015 kg kg$^{-1}$ while WRF simulations show higher values, around 0.0175 kg kg$^{-1}$. Moreover, despite seeming tiny, the variations in WRF simulations of Q near the surface cannot be overlooked as they might influence the amount of total accumulated precipitation. \vskip 0.2cm
Fig.\ref{WTHQ}b illustrates the vertical profiles temporal evolution of W, TH and Q of ERA5. The heatmap representing W shows an evident peak in intensity at 200 hPa fading going towards the surface and mainly occurring on July 20 and 21, which correspond to the day before and the day of the precipitation peak. During the rest of the event, slight alterations of W depend on the diurnal cycle. At night, values tend to be lower or negative, namely indicating downward direction, whereas during the day they tend to be positive. TH heatmap does not show any variation throughout the event, remaining constant the whole time. On the contrary, Q heatmap shows that during precipitation peak the moisture reaches higher pressure levels in the atmosphere, up to 400 hPa, whereas before the peak it reaches only up to 650 hPa.  \vskip 0.2cm
\begin{figure}[t!]\centering\includegraphics[width=1\textwidth]{./resultsMCS/W_wrf}\caption{ Vertical profile time evolution W by WRF (averaged over latitudes 26-35°N and longitudes 100-115°E). } \label{Wwrf} \end{figure}
\begin{figure}[t!]\centering\includegraphics[width=1\textwidth]{./resultsMCS/Qcloud}\caption{Qcloud vertical profile temporal evolution (averaged over latitudes 26-35°N and longitudes 100-115°E) and vertical profile (averaged over July 16-24, latitudes 26-35°N and longitudes 100-115°E).} \label{Qcloud} \end{figure}
The vertical profiles temporal evolution of W according to WRF simulations is presented in Fig.\ref{Wwrf}. The equivalent heatmaps for TH and Q are not shown as their differences are negligible and they are virtually identical to the heatmaps presented for ERA5 (Fig.\ref{WTHQ}b). Observing Fig.\ref{Wwrf}, we can see that the intensity and temporal evolution of W follows the same trend as precipitation. In every WRF run, the maximum updraft is simulated around July 20 and 21, with values ranging from 0.03 to above 0.04 m/s.  Thompson, Morrison and Ylin display more intense and narrow W peaks ranging from 200 hPa to 500 hPa, while CAM, WDM7 and WDM6 exhibit weaker peaks with varying vertical distributions. Compared to ERA5, all the MP schemes anticipate the peak of about one day, but the most similar by intensity and vertical extent are Thompson, Morrison and Ylin. It is also worth specifying that W values are of the order of magnitude of 10$^{-2}$ m/s because they are averaged over the horizontal 2D space, which is characterised by W$\sim$0 almost everywhere. In fact, analysing the original W without averaging over space, we find values in the expected range 1-10 m/s, typical of convective events.

\subsubsection{Non-precipitable hydrometeors}%2.5
\begin{figure}[t!]\centering\includegraphics[width=1\textwidth]{./resultsMCS/Qice}\caption{Qice vertical profile temporal evolution (averaged over latitudes 26-35°N and longitudes 100-115°E) and vertical profile (averaged over July 16-24, latitudes 26-35°N and longitudes 100-115°E).} \label{Qice} \end{figure}
In this section we will present the analysis of Qcloud and Qice, namely the mixing ratios of non-precipitable hydrometeors representing respectively the liquid and ice water contents in clouds. In Fig.\ref{Qcloud} the temporal evolution of the vertical profile and  the vertical profile averaged over time, latitude and longitude of Qcloud are illustrated. From the graph, we can observe that liquid water inside of clouds is found from 400 hPa downwards. Above that pressure level, Qcloud is virtually null. In ERA5, Qcloud vertical distribution exhibits two maxima, one of $\sim$3$\cdot$10$^{-5}$ kg kg$^{-1}$ at 500 hPa and another one of $\sim$3.4$\cdot$10$^{-5}$ kg kg$^{-1}$ at 800 hPa. In WRF simulations, both the vertical distribution and the magnitude depend on the MP scheme. Thompson, Morrison, Ylin and CAM display two peaks of Qcloud even though they vary in intensity, while WDM7 and WDM6 only show a single peak of greater intensity. The temporal evolution of Qcloud shows that also the time series changes using different MP schemes. In all datasets, there are clear diurnal cycles in which the variable seems higher during night and lower during the day. Thompson, Morrison and Ylin have a limited peak in time that appears during July 20 and 21, showing a similar behaviour to their precipitation time evolution, whereas CAM, WDM7 and WDM6 simulate peaks in Qcloud more distributed in time, mirroring precipitation behaviour represented by these MP schemes. Interestingly, ERA5 exhibits high Qcloud values throughout the whole event, in contrast with the narrow and intense precipitation peak. \vskip 0.2cm
Fig.\ref{Qice} presents the same analysis done for Qcloud but considering Qice. The vertical profile graph illustrates how vastly different the simulation of this variable is depending on the datasets. ERA5 shows an intense Qice peak of $\sim$2$\cdot$10$^{-5}$ kg kg$^{-1}$ at 300 hPa, while WRF runs present the same peak much weaker. Morrison is the most similar in terms of intensity, while the rest of the MP schemes simulates a maximum of magnitude roughly half or less than ERA5. Another interesting feature is the extent of the vertical distribution: while for ERA5 the presence of Qice is non negligible between 200 hPa and 600 hPa, this is not the case for all the other datasets. Thompson, Morrison, CAM and Ylin present non null Qice values only between 200 hPa and 300 hPa, demonstrating a much more concentrated vertical distribution of ice content, whereas WDM7 and WDM6 show a similar vertical distribution to ERA5 even though the intensity is lower. This behaviour is even more visible from the heatmaps. In particular, we can see how some schemes - Thompson and Ylin - simulate small values of Qice and in a very limited portion of the vertical profile, while others simulate Qice distributed in a thicker portion of atmosphere. Concerning the time evolution, the broadness and timing of the peak greatly vary in different schemes, but does not seem to have a specific relationship with precipitation. In fact, while it is true that the peak of Qice appears around the same time of maximum precipitation, it is not true that a broader Qice peak in time corresponds to a longer precipitation peak. \vskip 0.2cm
\begin{figure}[htbp]\centering\includegraphics[width=0.98\textwidth]{./resultsMCS/Qrain}\caption{Qrain vertical profile temporal evolution (averaged over latitudes 26-35°N and longitudes 100-115°E) and vertical profile (averaged over July 16-24, latitudes 26-35°N and longitudes 100-115°E).} \label{Qrain}\bigskip
\includegraphics[width=0.98\textwidth]{./resultsMCS/Qsnow}\caption{Qsnow vertical profile temporal evolution (averaged over latitudes 26-35°N and longitudes 100-115°E) and vertical profile (averaged over July 16-24, latitudes 26-35°N and longitudes 100-115°E).} \label{Qsnow} \end{figure}


\subsubsection{Precipitable hydrometeors}%1
\begin{figure}[t!]\centering\includegraphics[width=0.72\textwidth]{./resultsMCS/Qgraupel}\caption{Qgraupel vertical profile temporal evolution (averaged over latitudes 26-35°N and longitudes 100-115°E) and vertical profile (averaged over July 16-24, latitudes 26-35°N and longitudes 100-115°E).} \label{Qgraupel} \end{figure}
To complete our analysis, it is essential to consider also precipitable hydrometeors since they play a key role in precipitation formation processes. Fig.\ref{Qrain} illustrates Qrain vertical profile and its temporal evolution. The vertical profile graph suggests that all WRF simulations overestimate Qrain values. In fact, ERA5 displays an almost constant Qrain value of $\sim$1$\cdot$10$^{-5}$ kg kg$^{-1}$ from 600 hPa downward, while the majority of the schemes presents values two times higher. CAM stands out as it simulates values six times higher than ERA5 and three times higher than the average of the other MP schemes. Qrain time evolution agrees with precipitation time series. In particular, MP schemes that display higher Qrain values for limited time mirror the same timing of precipitation, while schemes that simulate longer Qrain maxima also have longer precipitation peaks in time. In all datasets, the timing of maximum Qrain corresponds to maximum precipitation. \vskip 0.2cm
Qsnow and Qgraupel vertical profiles and time evolutions are presented in Fig.\ref{Qsnow} and Fig.\ref{Qgraupel} respectively. In both cases non null values are found above 500 hPa, but the two parameters display different temporal evolutions. Compared to ERA5, Qsnow is overestimated in all WRF runs except in WDM7 and WDM6, where it is underestimated. Ylin displays the highest value of 0.00011 kg kg$^{-1}$ at 300 hPa, ERA5 peak is 0.0004 kg kg$^{-1}$ at 500 hPa and WDM7 produces the lowest peak of 1$\cdot$10$^{-5}$ kg kg$^{-1}$ at 400 hPa. Generally, MP schemes with longer peaks in time of Qsnow show a longer peak in precipitation, while shorter and more intense Qsnow peaks correspond to narrow and high precipitation peaks. Qgraupel is not simulated by ERA5, CAM and Ylin, but the majority of the other schemes presents a similar profile. WDM6 stands out for its Qgraupel intensity, which is roughly double the values of the other WRF runs with $\sim$1.6$\cdot$10$^{-5}$ kg kg$^{-1}$ at 500 hPa, while WDM7 stands out for the weakness of its peak, which is roughly one fourth compared to WDM6. The time evolution of Qgraupel is quite similar in all schemes, presenting maximum values from one day before to the day of precipitation peak. \vskip 0.2cm
We also analysed Qhail since it could have an impact on precipitation simulation of WDM7. The vertical profile displays a maximum at 500 hPa equal to $\sim$4.5$\cdot$10$^{-6}$ kg kg$^{-1}$, while the time evolution shows that the variable reacheas its highest values during precipitation peak (Fig.\ref{Qhail}). Therefore, Qhail likely contributes to precipitation formation in WDM7. 


\subsubsection{Liquid and ice water paths}%1.5
\begin{figure}[htbp]\centering\includegraphics[width=0.92\textwidth]{./resultsMCS/LLWP}\caption{LWP spatial distribution (averaged over July 16-24) and time evolution (averaged over latitudes 26-35°N and longitudes 100-115°E).} \label{LWP} 
\includegraphics[width=0.92\textwidth]{./resultsMCS/IIIWP}\caption{IWP spatial distribution (averaged over July 16-24) and time evolution (averaged over latitudes 26-35°N and longitudes 100-115°E).} \label{IWP} \end{figure}
LWP and IWP represent respectively the total amount of liquid- and solid-phase water content in the vertical direction. Fig.\ref{LWP} shows the time series and the spatial distribution of LWP. The time evolution indicates a major peak on July 21 in ERA5 and July 20 in WRF simulations, which correspond to precipitation maxima. Almost all MP schemes tend to well reproduce or slightly underestimate LWP compared to ERA5. CAM overestimates LWP with a peak of $\sim$0.8 kg m$^{-2}$, while the other schemes and ERA5 present peaks ranging from 0.25 kg m$^{-2}$ to 0.40 kg m$^{-2}$. The spatial patterns are in agreement with precipitation patterns, with peaks in LWP occurring in the same locations as precipitation peaks. Interestingly, the value of LWP is not proportionate to the intensity of P: it is not true that the MP schemes with higher LWP present the most intense precipitation. The peculiar results for CAM are caused by excessive Qrain values, while Qcloud is in line with other simulations. \vskip 0.2cm
In Fig.\ref{IWP} the spatial patterns and time evolution of IWP are presented. The main contributor to IWP is Qsnow, as its values are about one order of magnitude larger than those of Qice and Qgraupel. ERA5 time series displays one main peak in IWP magnitude of $\sim$0.40 kg m$^{-2}$ on July 21. Most WRF simulations show a peak on July 20 and 21, with values ranging from 0.2 kg m$^{-2}$ to 1 kg m$^{-2}$. Ylin and Thompson stand out because their IWP is higher compared to other datasets, with peak values of 1 kg m$^{-2}$ and 0.6 kg m$^{-2}$ respectively; while CAM, WDM7 and WDM6 produce a broader and weaker IWP peak, which is roughly half of ERA5. In particular, Ylin IWP high value could explain why OLR in this scheme is very low. IWP spatial patterns seems to reproduce precipitation patterns, especially for Thompson, Morrison, Ylin and ERA5. In CAM, WDM7 and WDM6 this relationship is less evident, as the magnitude of IWP is smaller, but their peaks are more broadly distributed in space as the correspondent patterns of precipitation. \vskip 0.2cm
\begin{table}[t!]\caption{Partial correlation coefficients of precipitation rate temporal evolution with LWP and IWP. The variables were standardised beforehand to limit effects due to their magnitude. For all the coefficients p-value $\le$0.0004.} \centering\includegraphics[width=1\textwidth]{./resultsMCS/tablecorrnew}\label{corrs} \end{table}
Table \ref{corrs} reports the partial correlation coefficients of the temporal evolution of precipitation rate and LWP, IWP. The vastly different values suggest that the impacts of liquid and solid water contents on precipitation vary depending on MP scheme. For some schemes, such as Thompson, Morrison and Ylin, the correlation between LWP and precipitation rate is very high, while the correlation with IWP is lower: in these cases, the driving factor behind precipitation seems to be the content of liquid water. On the contrary, for the remaining MP schemes and ERA5, the content of solid hydrometeors seems to be dominant. In CAM, the two correlation coefficients are close, which indicates that liquid and solid hydrometeors contribute similarly to precipitation outcome. It is interesting to note that, despite being relatively low in WDM7 and WDM6, IWP seems to be the driving factor behind precipitation in these two MP schemes. 


\subsubsection{Cloud fraction}%1
\begin{figure}[t!]\centering\includegraphics[width=1\textwidth]{./resultsMCS/cldfra}\caption{Cloud fraction vertical profile temporal evolution (averaged over latitudes 26-35°N and longitudes 100-115°E) and vertical profile (averaged over July 16-24, latitudes 26-35°N and longitudes 100-115°E).} \label{cldfra} \end{figure}
Fig.\ref{cldfra} illustrates the cloud fraction vertical profile and its time evolution. According to ERA5, the cloud fraction has three major peaks at about 200 hPa, 600 hPa and 800 hPa of 0.27, 0.10 and 0.12 respectively. CAM gives the most similar representation, with peaks of 0.16, 0.12 and 0.15 roughly at the same pressure levels as ERA5. The other MP schemes produce vastly different profiles. Thompson, Morrison and Ylin display a major peak at 200 hPa of 0.24, 0.35 and 0.40 respectively, and a minor peak at 800 hPa of 0.05. WDM7 and WDM6 show three peaks at the same pressure levels but generally less intense than ERA5. It is interesting to note that, while ERA5 and Thompson have cloud fraction values non null also very close to the surface, the cloud fraction in other schemes becomes virtually zero at 975 hPa. The heatmaps show how the datasets that simulate an intense and narrow precipitation peak produce also a maximum of cloud fraction in the same days, namely Thompson, Morrison, Ylin and ERA5, whereas CAM, WDM7 and WDM6 exhibit a broader maximum in time for cloud fraction, that mirrors the time evolution of precipitation in those MP schemes. 
%The comparison of cloud fraction with Qcloud and Qice vertical distributions leads to some interesting considerations. Since Qice is found mainly above 500 hPa and Qcloud below 400 hPa, the cloud fraction peak at 200 hPa must be related to the presence of ice-phase water. However, if we look at WRF runs with the most intense peak at 200 hPa, namely Ylin, Thompson and Morrison, their Qice vertical distribution is not necessarily in agreement with the cloud fraction value. In fact, even though Morrison has relatively high values of Qice around 200 hPa, Ylin and Thompson Qice values at that pressure level are not nearly as intense and do not justify the peak in cloud fraction. On the other hand, schemes that have a thick cloud layer with less pronounced peaks, such as WDM7 and WDM6, also show a similar distribution for Qice. The minor peak slightly above 600 hPa present in ERA5, CAM, WDM7 and WDM6 could be composed mainly by mixed-phase clouds, since for the majority of the datasets it coincides with the overlap of Qice and Qcloud distributions. Below 600 hPa, the only factor that could explain the presence of clouds would be the presence of liquid water. This result seems reasonable because ERA5 and CAM, which have the strongest peak of cloud fraction at $\sim$800 hPa, also display the highest Qcloud values at that level. 

\subsubsection{Horizontal winds}%1
\begin{figure}[t!] \label{windslat}\centering\includegraphics[width=1\textwidth]{./resultsMCS/winds_latdiff}\caption{Vertical profile of northward winds differences of WRF runs minus ERA5 (averaged over longitudes 100-115°E and time July 16-24) and ERA5 vertical profile as reference.} \label{windslati} \end{figure}
Fig.\ref{windslati} illustrates the vertical profiles of northward wind differences of WRF runs minus ERA5. All MP schemes present the same patterns of anomalies but definitely different magnitudes. All WRF runs display two major regions of difference from ERA5: at 200 hPa and around 800 hPa. At the highest level, the negative values indicate that the winds directed towards south are weaker in WRF simulations. This is especially true for CAM, WDM7 and WDM6 that exhibit negative anomalies of about -3 m/s, which represents roughly 50\% difference from the absolute value of the ERA5 wind. At 800 hPa, the anomalies are quite similar in all MP schemes with values around -2 m/s, differing by $\sim$80\% from ERA5 values, which means that in WRF simulations the wind towards north at this level is much weaker than in ERA5. Relevant differences also occurr in the simulation of eastward winds (Fig.\ref{windslon}). \vskip 0.2cm
These anomalies could contribute to the precipitation process in multiple ways. Firstly, the variations in winds intensity could cause alterations in precipitation location. In all WRF runs, winds in the lower portion of atmosphere are directed towards north but with less intensity than ERA5. This could provide an explanation for the observed southward shift in precipitation peak in WRF runs compared to ERA5. Secondly, the stronger contrast shown at different atmospheric levels in some MP schemes could influence the development of the convective system, possibly contrasting its formation and leading to less and more broadly distributed precipitation.



%%%%%%%%%%%%%%%%%%%%%%%%%%%%%%%%%%%%%%%%%%%%%%%%%%%%%%%%%%%%%%%%%%%%%%%%%%


\section{Heavy snow system case}
The following section illustrates the results of the same analysis carried out for the MCS but considering the heavy snow system. Exactly like in the former analysis, we will consider first precipitation, OLR and T2m, then other 2D variables related to precipitation processes and lastly the vertical profiles of 3D variables, including W, hydrometeors, cloud fraction and horizontal winds.

\subsection{Precipitation, outgoing longwave radiation and temperature}%tot 10
\subsubsection{Temporal evolution} %2
The time series of precipitation rate, T2m, OLR and accumulated precipitation are presented in Fig.\ref{SPPOLRT}. As we can observe, all datasets simulate coherenly the timing of precipitation peak on October 8, while its intensity displays major differences. In fact, ERA5 overerestimates the precipitation maximum of roughly 2 mm/day compared to IMERG, which is around 3.5 mm/day.
\begin{figure}[t!]\centering\includegraphics[width=0.98\textwidth]{./resultsSNOW/PPOLRT}\caption{Time series averaged over spatial domain (27-35°N and 90-105°E) of a) Daily mean precipitation rate, b) Daily mean T2m, c) Daily mean OLR and d) Hourly accumulated precipitation. The dashed lines identify validation datasets, while the solid lines are used for WRF runs.} \label{SPPOLRT} \end{figure}
\begin{figure}[t!]\centering\includegraphics[width=0.92\textwidth]{./resultsSNOW/corrs}\caption{Correlograms of hourly temporal evolution averaged over spatial domain (27-35°N and 90-105°E) of a) precipitation rate, b) T2m, c) OLR and d) Accumulated precipitation.} \label{SCORR} \end{figure}
Precipitation rate shows also a minor peak on October 5, which is present in all datasets except IMERG. WDM7 and WDM6 reproduce precipitation intensity very similarly to ERA5, while the rest of the schemes - Thompson, Morrison and Ylin - overestimate the major peak of about 2 mm/day. T2m is reproduced with the same decreasing trend among WRF schemes, with differences in intensity always below 1°C. Despite having a similar time evolution, WRF exhibits values roughly 1.5°C lower than ERA5 throughout the whole event. OLR trend  is similar between datasets and shows minima on the same day of precipitation peak: sinks of ERA5, WDM7, WDM6 and Thompson are around 210 W m$^{-2}$, while Morrison and Ylin display lower values at 195 W m$^{-2}$ and 185 W m$^{-2}$. Similarly to the previous study case, we observe that Ylin presents the lowest OLR of all. Lastly, the accumulated precipitation reflects what has been said for precipitation rate: ERA5, WDM7 and WDM6 display values around 20 mm, IMERG underestimates it with a total accumulated precipitation of 10 mm, whereas the rest of the schemes present values $\sim$30 mm. \vskip 0.2cm
\begin{figure}[t!]\centering\includegraphics[width=0.98\textwidth]{./resultsSNOW/POLR_maps}\caption{Spatial distribution averaged over time (October 4-9) of a) precipitation, b) OLR.} \label{SPOLRmaps}
\centering\includegraphics[width=0.98\textwidth]{./resultsSNOW/POLR-eraimerg}\caption{Spatial distribution averaged over time (October 4-9) of a) precipitation according to ERA5 and IMERG, b) OLR according to ERA5.} \label{SPOLRmapsERA}\end{figure}
Fig.\ref{SCORR} illustrates the correlograms of the temporal evolution of precipitation rate, T2m, OLR and accumulated precipitation. The precipitation correlogram shows that the MP scheme giving the most similar precipitation rate to IMERG and ERA5 are WDM6 and Morrison, with coefficients of 0.85 and 0.82 respectively; while Ylin and WDM7 are the most different, with correlation coefficients of 0.64 and 0.73. The correlation between T2m datasets is very high, with the lowest value being 0.95 between ERA5 and WDM6. The remaining coefficients, both relative to WRF runs and ERA5, indicate that T2m time evolution is simulated very similarly in all datasets. OLR correlogram demonstrates overall high values but also shows that Ylin reproduces OLR with some variations compared to other schemes. Similarly to the MCS case, the performance of WRF substantially changes when removing the diurnal cycle from T2m and OLR (Fig.\ref{decy2}), suggesting that hourly oscillations of the two variables are simulated differently depending on the MP scheme. Lastly, accumulated precipitation illustrates extremely high coefficients among the schemes and also compared to ERA5 and IMERG, which means that the overall time evolution of accumulated precipitation is coherently simulated in all datasets.

\subsubsection{Spatial patterns} %3
The spatial distributions of precipitation and OLR are presented in Fig.\ref{SPOLRmaps}. WRF simulations display precipitation peaks around 29°N and 94°E, with intensity depending on the MP scheme. Thompson, Morrison and Ylin show higher and more distributed values, while WDM7 and WDM6 simulate smaller and weaker peaks.
In contrast with the previous study case, OLR patterns are not in agreement with precipitation maxima. In fact, it is clear that they are heavily influenced by topography (with lower OLR over high altitudes) and the lowest OLR values do not coincide with precipitation peaks. \vskip 0.2cm
\begin{figure}[t!]\centering\includegraphics[width=1\textwidth]{./resultsSNOW/diffrimergP}\caption{Differences in precipitation spatial distribution (averaged over October 4-9) using IMERG as reference.} \label{SPdiff} \end{figure}
Comparing Fig.\ref{SPOLRmaps} and Fig.\ref{SPOLRmapsERA}, differences in intensity and spatial distributions are observable. In fact, ERA5 presents a precipitation peak around 28°N and 94°E with limited dimensions, while the majority of WRF runs produce stronger and larger peaks. WDM7 and WDM6 are the schemes most resembling ERA5, both in terms of precipitation intensity and spatial extent. It is also interesting to note how IMERG does not show the main precipitation peak in the southwest area, but only produces mild precipitation over the northeast area of the domain. This is in agreement with its timeline, which shows that IMERG illustrates the lowest occurred precipitation. Discrepancies in spatial patterns and intensity of precipitation are even more clear in Fig.\ref{SPdiff}, where IMERG is used as reference. Observing the location of the strong positive in WRF runs compared to IMERG, we can conclude that WRF runs tend to reproduce a precipitation peak which does not appear in IMERG. Moreover, by comparing the location of strong positive biases in WRF simulations and ERA5, we can conclude that WRF runs tend to reproduce precipitation peaks slightly more north than ERA5. \vskip 0.2cm
Similarly to the previous case study, MP schemes are proved to substantially influence the intensity and spatial patterns of precipitation simulations. However, in this case the timing of precipitation is coherent among all datasets. Another difference with the previous case study concerns OLR, which does not accurately indicate where precipitation occurred since areas with lower OLR do not correspond to areas with intense precipitation. 


\subsection{Analysis of other 2D variables}%5
\subsubsection{Convective available potential energy}
\begin{figure}[t!]\centering\includegraphics[width=0.95\textwidth]{./resultsSNOW/CAPE}\caption{CAPE spatial distribution averaged over time (October 4-9) and daily time evolution averaged over spatial domain (27-35°N and 90-105°E).} \label{Scape} \end{figure}
In Fig.\ref{Scape} the spatial distribution and time series of CAPE are presented. The temporal evolution of WRF runs displays a steady trend, mantaining values between 10-25 J kg$^{-1}$ throughout the whole event. Thompson, Morrison and Ylin simulate the highest CAPE, while WDM7 and WDM6 show lower values. WRF runs exhibit values three to four times smaller than ERA5, which peaks at 70 J kg$^{-1}$ on October 7. The spatial patterns are similar among all datasets and differ mostly by intensity. However, ERA5 shows very high values of CAPE at the western limit of the domain, which are not reproduced by WRF runs. In all datasets, the spatial distributions reflect the topography of the area: in particular, high CAPE values correspond to low elevation areas. Nonetheless, if compared to the MCS case, CAPE is 10 times smaller. Moreover, there is no clear relationship between CAPE values and precipitation, as it is not true that higher CAPE leads to higher precipitation or viceversa. 
\begin{figure}[t!]\centering\includegraphics[width=0.95\textwidth]{./resultsSNOW/PBLH}\caption{PBLH spatial distribution averaged over time (October 4-9) and daily time evolution averaged over spatial domain (27-35°N and 90-105°E).} \label{Spblh} \end{figure}


\subsubsection{Planetary boundary layer height}
The spatial patterns and temporal evolutions of PBLH are presented in Fig.\ref{Spblh}. Both in WRF simulations and ERA5, PBLH reaches its maximum one day before precipitation peak and then slowly decreases. WRF runs show similar trends to that of ERA5, but with values consistently bigger by 150-200 m. In fact, ERA5 maximum PBLH is around 370 m, Morrison and Ylin produce the highest PBLH at $\sim$500 m, while Thompson, WDM7 and WDM6 simulate maxima at $\sim$450 m. Spatially, all datasets exhibit a similar pattern, with highest values in the high elevation areas of the southern half of the domain. There seems to be no direct temporal or spatial relationship between PBLH and precipitation, since the MP schemes with higher/lower PBLH do not simulate more or less precipitation. Compared to the MCS case, PBLH is simulated more coherently among WRF runs both in terms of intensity and time evolution. It is interesting to note how in this case WRF reproduces the time series similarly to ERA5 but with greater values.


\subsubsection{Moisture flux at surface}
\begin{figure}[t!]\centering\includegraphics[width=1\textwidth]{./resultsSNOW/moisture}\caption{Moisture flux at surface spatial distribution averaged over time (October 4-9) and daily time evolution averaged over spatial domain (27-35°N and 90-105°E).} \label{Sqflux} \end{figure}
Fig.\ref{Sqflux} displays the spatial distribution and time evolution of moisture flux at surface. The time evolution shows some trend variations depending on the dataset considered but all values are in the order of magnitude of 10$^{-5}$  kg m$^{-2}$s$^{-1}$ and are overall positive, indicating upward flux. ERA5, WDM7 and WDM6 display a slowly decreasing trend that becomes quicker after October 7. Thompson and Morrison show a constant decreasing trend that recheas its minimum of  $\sim$1.3$\cdot$10$^{-5}$ kg m$^{-2}$s$^{-1}$ when precipitation is at its peak and afterwards starts increasing again. Ylin initially displays a similar trend to Thompson and Morrison, while after the minimum is more similar to WDM6. It is interesting to note that MP schemes that display a trend resembling that of ERA5, namely WDM7 and WDM6, also show similar precipitation simulations. Spatially, the pattern of the moisture flux is heavily influenced by the topography of the region. Lower elevations and areas with water bodies are characterised by higher values of moisture flux, while higher elevations show weaker upward flux. All datasets display similar spatial patterns with upward moisture flux being more intense in the area where precipitation occurs. These results suggest that moisture flux at surface played a relevant role in precipitation processes in this case.


\subsection{Vertical profile of atmosphere}%tot 10
\subsubsection{Vertical velocity, potential temperature and specific humidity}%2.5
The vertical profiles of W, TH and Q are illustrated in Fig.\ref{SWTHQ}a. The vertical profile of W shows major variations among datasets both in intensity and vertical extent. Moreover, values are mostly positive, indicating an overall upward convection. ERA5 shows a single peak of about 0.011 m/s at 300 hPa, while all WRF runs simulate an utterly different vertical profile with two major peaks: one at 300 hPa and the other at 700 hPa. Ylin, Thompson and Morrison produce the highest maxima, with values around 0.015 m/s and 0.02 m/s for the two peaks. WDM7 and WDM6 exhibit slightly weaker peaks, with values of $\sim$0.012 m/s and $\sim$0.014 m/s. The whole vertical profile of TH is extremely coherent among all datasets, with values ranging from 350 K at 200 hPa to 300 K at 875 hPa. Lastly, Q is reproduced very similarly among all datasets up to 600 hPa, where ERA5 and WRF simulations diverge. In fact, near the Earth's surface, ERA5 exhibits a value of $\sim$0.005 kg kg$^{-1}$ while WRF simulations show values two times bigger. As for the previous case study, the differences in Q between WRF runs are very small but should not be ignored as they could influence the amount of total accumulated precipitation. \vskip 0.2cm
Fig.\ref{SWTHQ}b represents the vertical profiles temporal evolution of W, TH and Q of ERA5. The heatmap of W shows a clear peak in intensity of $\sim$0.025 m/s at 300 hPa fading going downwards and occurring on October 7 and 8, namely the day before and the day of the precipitation peak. During the rest of the event, small alterations of W depend on the diurnal cycle. At night values tend to be lower or negative, indicating downward direction, whereas during the day they are mostly positive. TH heatmap does not show any variation throughout the event, while Q heatmap exhibits a decrease in values near the surface after the precipitation peak on October 8.  \vskip 0.2cm
\begin{figure}[t!]\centering\includegraphics[width=1\textwidth]{./resultsSNOW/WTHQ}\caption{a) Vertical profile of W, TH and Q (from left to right) averaged over October 4-9, latitudes 27-35°N and longitudes 90-105°E. b) Vertical profile time evolution of ERA5 W, TH and Q (averaged over latitudes 27-35°N and longitudes 90-105°E).} \label{SWTHQ} \centering\includegraphics[width=1\textwidth]{./resultsSNOW/W_wrf}\caption{ Vertical profile time evolution W by WRF (averaged over latitudes 27-35°N and longitudes 90-105°E). } \label{SWwrf} \end{figure}
The temporal evolution of W vertical profiles simulated by WRF is presented in Fig.\ref{SWwrf}. The equivalent heatmaps for TH and Q are not shown as their differences are negligible and they are virtually identical to the heatmaps presented for ERA5 (Fig.\ref{SWTHQ}b). Overall, the intensity and temporal evolution of W (Fig.\ref{SWwrf}) follow the same trend as precipitation. In WRF runs, the maximum updraft is simulated on October 7 and 8, with values ranging from 0.03 m/s to above 0.04 m/s.  Thompson, Morrison and Ylin display more intense W peaks ranging from 200 hPa to 800 hPa, while WDM7 and WDM6 exhibit peaks with the same vertical distribution but smaller magnitudes. Compared to ERA5, WRF runs simulate a stronger peak at 300 hPa and an additional peak at 700 hPa of even higher intensity, whereas the timing is quite similar. Because of the intensity, WDM7 and WDM6 are the closest MP schemes to ERA5. Similarly to the previous case study, W values are of the order of magnitude of 10$^{-2}$ m/s because they are averaged over the horizontal 2D space, which is characterised by W$\sim$0 almost everywhere. By observing the original W without averaging over space, values are in the expected range 1-10 m/s, typical of convective events.

\subsubsection{Non-precipitable hydrometeors}%2.5
\begin{figure}[t!]\centering\includegraphics[width=1\textwidth]{./resultsSNOW/Qcloud}\caption{Qcloud vertical profile temporal evolution (averaged over latitudes 27-35°N and longitudes 90-105°E) and vertical profile (averaged over October 4-9, latitudes 27-35°N and longitudes 90-105°E).} \label{SQcloud} \end{figure}
\begin{figure}[t!]\centering\includegraphics[width=1\textwidth]{./resultsSNOW/Qice}\caption{Qice vertical profile temporal evolution (averaged over latitudes 27-35°N and longitudes 90-105°E) and vertical profile (averaged over October 4-9, latitudes 27-35°N and longitudes 90-105°E).} \label{SQice} \end{figure}
The time evolution of the vertical profile and the vertical profile averaged over time, latitude and longitude of Qcloud are illustrated in Fig.\ref{SQcloud}. The values of Qcloud are virtually null above 400 hPa, while below that pressure level they are of the order of magnitude of 10$^{-5}$ kg kg$^{-1}$. The biggest Qcloud values are found at 700 hPa in all datasets, but the intensity of the peaks vary. ERA5 shows a peak of about 0.00006 kg kg$^{-1}$ and the most similar MP scheme is Ylin, with a maximum of 0.00005 kg kg$^{-1}$. The rest of the schemes overestimates the peak, with values ranging from 0.00008 kg kg$^{-1}$ in Morrison to 0.00012 in WDM7 and WDM6. The temporal evolution of Qcloud also depends on the MP scheme, but overall we can observe steady Qcloud values before the precipitation peak and a noticeable decrease afterwards. On the contrary, ERA5 produces steady values of Qcloud followed by an increase after the precipitation peak. There seems to be no direct relationship with precipitation, as the days of maximum precipitation intensity do not display any evident variation in Qcloud. \vskip 0.2cm
Fig.\ref{SQice} presents the same analysis but considering Qice and demonstrates how vastly different the representation of this variable is in our datasets. ERA5, WDM7 and WDM6 exhibit an intense Qice peak of $\sim$1.4$\cdot$10$^{-5}$ kg kg$^{-1}$ around 400 hPa, while the other WRF runs present weaker peaks. Morrison simulates a maximum of $\sim$1.2$\cdot$10$^{-5}$ kg kg$^{-1}$ at a similar pressure level, while Ylin and Thompson drastically underestimate Qice with values of $\sim$0.3$\cdot$10$^{-5}$ kg kg$^{-1}$ and  $\sim$0.15$\cdot$10$^{-5}$ kg kg$^{-1}$ respectively. The extent of the vertical distribution is another feature that differs substantially: while in ERA5 the presence of Qice is non negligible between 200 hPa and 600 hPa, this is not the case in all the other datasets. In fact, Thompson, Morrison and Ylin present non null Qice values only between 200 hPa and 400 hPa, whereas WDM7 and WDM6 show a vertical distribution extending from 300 hPa to 800 hPa. Moreover, the time evolution, both in ERA5 and WRF simulations, seems to be in agreement with precipitation evolution, as Qice increases during precipitation maximum. 
\begin{figure}[htbp]\centering\includegraphics[width=0.955\textwidth]{./resultsSNOW/Qrain}\caption{Qrain vertical profile temporal evolution (averaged over latitudes 27-35°N and longitudes 90-105°E) and vertical profile (averaged over October 4-9, latitudes 27-35°N and longitudes 90-105°E).} \label{SQrain}\bigskip
\includegraphics[width=0.955\textwidth]{./resultsSNOW/Qsnow}\caption{Qsnow vertical profile temporal evolution (averaged over latitudes 27-35°N and longitudes 90-105°E) and vertical profile (averaged over October 4-9, latitudes 27-35°N and longitudes 90-105°E).} \label{SQsnow} \end{figure}



\subsubsection{Precipitable hydrometeors}%1
In Fig.\ref{SQrain} Qrain vertical profiles and temporal evolutions are illustrated. The vertical profile graph shows that all MP schemes overestimate this variable. In fact, ERA5 displays an almost constant Qrain value of $\sim$0.25$\cdot$10$^{-5}$ kg kg$^{-1}$ from 700 hPa downward, while the majority of the schemes presents much higher values, ranging from 1.5$\cdot$10$^{-5}$ to 2.5$\cdot$10$^{-5}$ kg kg$^{-1}$. The time evolution agrees with precipitation time series in most datasets. In particular, Thompson, WDM7, WDM6 and ERA5 display a clear intensification of Qrain coinciding with precipitation peak, whereas the other MP schemes do not simulate a clear peak. \vskip 0.2cm
\begin{figure}[t!]\centering\includegraphics[width=0.7\textwidth]{./resultsSNOW/Qgraupel2}\caption{Qgraupel vertical profile temporal evolution (averaged over latitudes 27-35°N and longitudes 90-105°E) and vertical profile (averaged over October 4-9, latitudes 27-35°N and longitudes 90-105°E).} \label{SQgraupel} \end{figure}
Qsnow vertical profiles and time evolutions are presented in Fig.\ref{SQsnow}. The vertical profile graph suggests that Qsnow is found between 200 hPa and 700 hPa in WRF simulations, while in ERA5 it extends all the way to 875 hPa. The intensity of Qsnow maxima varies in different datasets: ERA5, WDM7 and WDM6 produce a peak of about 0.00002 kg kg$^{-1}$, whereas the rest of the schemes overestimate it with values ranging from 0.00005 kg kg$^{-1}$ to 0.0001 kg kg$^{-1}$. The time evolution of Qsnow mirrors the one of precipitation: higher values of Qsnow correspond to the days of precipitation peak. It is also interesting to note that in this case there is a relationship between the magnitude of the variable and the intensity of P: datasets with lower Qsnow, namely ERA5, WDM7 and WDM6, also produce lower precipitation. \vskip 0.2cm
Fig.\ref{SQgraupel} shows Qgraupel vertical profiles and temporal evolutions. Qgraupel is not simulated by ERA5 and Ylin, but the majority of the other schemes presents a similar profile, with non null values extending from 400 hPa to 800 hPa. WDM6 stands out for its Qgraupel intensity, which is roughly double the one of other WRF runs with $\sim$5$\cdot$10$^{-5}$ kg kg$^{-1}$ at 700 hPa, while Morrison has the weakest peak of $\sim$1.8$\cdot$10$^{-5}$ kg kg$^{-1}$. The time evolution of Qgraupel is quite similar in WDM7 and WDM6, presenting maximum values from one day before to the day of precipitation peak. \vskip 0.2cm
Despite being simulated only by one MP scheme - WDM7 - we also analysed Qhail because it could have an impact on its precipitation outcome (Fig.\ref{SQhail}). The vertical profile shows a peak at 600 hPa equal to $\sim$4.5$\cdot$10$^{-6}$ kg kg$^{-1}$, while the time evolution shows that the variable recheas its highest values during precipitation peak. Therefore, for this MP scheme it is likely that Qhail takes part in precipitation formation. \vskip 0.2cm
Overall, in the heavy snow system case we see solid-phase hydrometeors dominating precipitation due to their magnitude and vertical extent. In fact, all hydrometeors reach pressure levels closer to the surface due to the lower temperatures during this event compared to the MCS case. Thus, compared to the previous case study, relevant precipitation formation processes occur lower in the atmosphere.
\begin{figure}[htbp]\centering\includegraphics[width=0.92\textwidth]{./resultsSNOW/LWP}\caption{LWP spatial distribution (averaged over October 4-9) and time evolution (averaged over latitudes 27-35°N and longitudes 90-105°E).} \label{SLWP} 
\includegraphics[width=0.92\textwidth]{./resultsSNOW/IWP}\caption{IWP spatial distribution (averaged over October 4-9) and time evolution (averaged over latitudes 27-35°N and longitudes 90-105°E).} \label{SIWP} \end{figure}

\subsubsection{Liquid and ice water paths}%1.5
Fig.\ref{SLWP} shows the time series and the spatial distribution of LWP. The time evolution completely differs depending on the dataset. ERA5 shows an increasing trend during the first day, followed by a plateau around 0.2 kg m$^{-2}$ that ends October 8, when LWP begins to increase again. On the contrary, WRF runs show a totally different trend: Thompson and Morrison show one major peak of $\sim$0.5 kg m$^{-2}$ on October 8, Ylin exhibits roughly constant values from October 5 to October 8, whereas WDM7 and WDM6 display two main peaks slightly above 0.4 kg m$^{-2}$ on October 5 and October 8. The spatial patterns of WRF simulations suggest that the majority of LWP is concentrated in the areas of lower elevation, especially over the southeast corner of the domain, but a minor peak is displayed in the area of precipitation peak. On the contrary, ERA5 produces non null values also in higher elevation regions. \vskip 0.2cm
In Fig.\ref{SIWP} the spatial patterns and time evolution of IWP are illustrated. Similarly to the MCS case, the main contributor to IWP is Qsnow. For IWP, the time evolution is quite similar in all datasets, with major differences in intensity. In fact, all datasets display a major IWP peak on October 8: ERA5, WDM7 and WDM6 exhibit maxima of about 0.2 kg m$^{-2}$, while Morrison, Thompson and Ylin show values of 0.3 kg m$^{-2}$, 0.4 kg m$^{-2}$ and 0.55 kg m$^{-2}$ respectively. Similarly to the previous study case, high values of IWP for Ylin could be the cause of its low OLR values. IWP spatial patterns agree with precipitation patterns, especially in ERA5, WDM7 and WDM6 where the only IWP peak coincides with the area of precipitation peak. In the other schemes, there are IWP maxima in the same areas of precipitation peak, but high values of IWP are shown also in the northeast area of the domain where mild and weak precipitation occurrs. \vskip 0.2cm
\begin{table}[t!]\caption{Partial correlation coefficients of precipitation rate temporal evolution with LWP and IWP. The variables were standardised beforehand to limit effects due to their magnitude. For all the coefficients p-value $\le$0.029.} \centering\includegraphics[width=1\textwidth]{./resultsSNOW/partialcorr}\label{Scorrs} \end{table}

Table \ref{Scorrs} reports the partial correlation coefficients of precipitation rate and LWP, IWP. For all datasets, IWP has a very high correlation with values always above 0.75, indicating that for this case study the solid-phase hydrometeors are driving precipitation formation processes. The correlation with LWP is in general much lower and in some cases negative. In fact, Thompson, Morrison and Ylin show correlation values of 0.26, 0.09 and 0.54 respectively, which suggest that liquid hydrometeors contribute only marginally to precipitation. Interestingly, ERA5, WDM7 and WDM6 produce low and negative correlation. This means that liquid hydrometeors only play a minor role in precipitation formation in these WRF simulations and their presence does not favour precipitation formation.

\subsubsection{Cloud fraction}%1
\begin{figure}[t!]\centering\includegraphics[width=1\textwidth]{./resultsSNOW/cldfra}\caption{Cloud fraction vertical profile temporal evolution (averaged over latitudes 27-35°N and longitudes 90-105°E) and vertical profile (averaged over October 4-9, latitudes 27-35°N and longitudes 90-105°E).} \label{Scldfra} \end{figure}
Fig.\ref{Scldfra} represents the vertical profile and time evolution of cloud fraction. The vertical profile of ERA5 illustrates two main peaks: one at 300 hPa of 0.22 and one at 700 hPa of 0.19. WRF simulations display very different vertical profiles compared to ERA5 and also between themselves. In fact, Thompson, Morrison and Ylin produce two similar peaks at 300 hPa and 500 hPa of $\sim$0.3 and $\sim$0.35 respectively, whereas WDM7 and WDM6 show only one main peak at 700 hPa of 0.3. The heatmaps show that there are no clear variations in cloud fraction corresponding to precipitation peak. In fact, both ERA5 and WRF simulations produce only slightly bigger values of cloud fraction on October 8, which decrease substantially afterwards. The major discrepancies found in cloud fraction simulations are likely linked to the representation of hydrometeors that compose them, therefore it would be interesting to further investigate the relationship between hydrometeors and cloud fraction.

%Comparing cloud fraction with Qcloud and Qice, we can draw some conclusions on the phase of the clouds based on their vertical location. Qice vertical extent and intensity is very different based on the dataset considered. Thompson and Ylin have very small values and very limited extent at 300 hPa, while Morrison is also limited to this level but with much higher Qice values. WDM7 and WDM6 have relatively high values of Qice extending from 300 hPa to 800 hPa, while ERA5 recheas only 600 hPa. The cloud fraction peak at 300 hPa in Morrison could be justified by its Qice, which has a limited vertical extent but high values, while it cannot be explained using Qice for Thompson and Ylin, which have extremely low values of Qice at that pressure level. Instead, for ERA5 the cloud fraction peak at 300 hPa is in agreement with its representation of Qice. Considering the cloud fraction peak at 500 hPa for Morrison, Thompson and Ylin, ... - does not have much sense if we dont consider all hydrometeors, but that's almost impossible to do. maybe delete this from here and also MCS.

\subsubsection{Horizontal winds}%1
In Fig.\ref{Swindslat} the vertical profiles of northward wind differences of WRF simulations minus ERA5 are illustrated. All MP schemes present the same patterns but with different magnitudes. All WRF runs exhibit a major positive difference of 2.5 m/s in winds directed towards north between 100 hPa and 500 hPa and another positive difference of about 1 m/s close to the surface, while ERA5 displays values around 3 m/s and 1 m/s respectively. This means that in WRF simulations, especially Thompson, Morrison and Ylin, winds directed towards north are more intense compared to ERA5 by roughly 80-90\%. These anomalies could alterate precipitation location and explain the shift towards north observed in WRF simulations compared to ERA5. Moreover, the more intense winds in Thompson, Morrison and Ylin could also contribute to enhance precipitation formation and explain why precipitation is substantially overestimated by these MP schemes. Relevant differences were also found in the simulation of eastward winds (Fig.\ref{Swindslon}).
%\makeatletter
%\setlength{\@fptop}{0pt}
%\makeatother
\begin{figure}[!t] \label{windslat}\centering\includegraphics[width=1\textwidth]{./resultsSNOW/windslat}\caption{Vertical profile of northward winds differences of WRF runs minus ERA5 (averaged over longitudes 90-105°E and time October 4-9) and ERA5 vertical profile as reference.} \label{Swindslat} \end{figure}







\chapter{Discussion}
\rhead{\textsl{Chapter 4: Discussion}}
\renewcommand{\headrulewidth}{0pt}
In this chapter we will look at the representation of atmospheric energy budget, moisture content, convection, cloud composition and discuss their effects on precipitation simulation. Moreover, we will take into accounts the impacts of MP schemes characteristics. Lastly, limitations of our analysis will be clarified and possibilities of expansion for future research will be presented. 
%The energy budget is discussed considering mainly OLR and CAPE; the moisture content using moisture flux at surface, Q and horizontal winds; convection using W, TH and PBLH; cloud composition will be discussed through hydrometeors mixing ratios, cloud fraction and LWP/IWP.


\section{Factors affecting precipitation}
\subsection{Energy budget}%2 pag
Generally, low OLR values at the top of atmosphere indicate deeper convection and are associated with higher cloud fraction \citep{Hazra2017}. Our results for the MCS case study agree with previous findings as the connection between OLR and convective precipitation is evident (Fig.\ref{PPOLRT} and Fig.\ref{POLRmaps}). %, however some major differences arise between datasets. Firstly, WRF OLR sinks anticipate by one or two days ERA5 minima, mirroring the timing of corresponding precipitation. Secondly, WRF runs that exhibit a broad and weak OLR sink - WDM7, WDM6 and CAM - reflect the same pattern in precipitation, indicating that the development of deep convection is more distributed in time and less strong. In the Morrison run, OLR is underestimated compared to ERA5, but the precipitation peak is proportionately high. On the contrary, in the Ylin run OLR is underestimated throughout the whole event without showing a very intense precipitation peak. 
In the heavy snow system case study, there is a correspondence between OLR and precipitation time evolutions but not spatial patterns. %This is an indication that convection might only have a minor role in precipitation formation during the heavy snow event. %Compared to validation datasets, the majority of MP schemes - Thompson, Morrison and Ylin - overestimates precipitation. Moreover, Ylin and Morrison experiments once again produce the lowest OLR values. 
%A recent study of the same heavy snow system confirms that Thompson overestimates precipitation amount but overall provides a good simulation of the event \citep{Lin2023}. 
In the analysis of radiation fluxes, it is important to remember that the cloud phase plays a major role, as clouds with high ice content can absorb more longwave radiation and thus reduce OLR \citep{Fu2011}. By comparing OLR and IWP values in both case studies (Fig.\ref{PPOLRT} and Fig.\ref{IWP} for the MCS case, Fig.\ref{SPPOLRT} and Fig.\ref{SIWP} for the heavy snow case), a relationship emerges between the two variables. MP simulations that produce high IWP - Ylin, Thompson and Morrison - display the lowest OLR values, both in terms of time evolution as well as spatial distribution. However, it has to be noted that the intensity of one variable is not necessarily proportional to the magnitude of the other: for instance, Thompson displays the second highest IWP but the third lowest OLR. This could be due to its representation of Qice, which is heavily underestimated compared to the rest of MP schemes and could contribute to the radiation feedbacks more than other solid-phase water species. A study conducted over the Artic also found that Morrison leads to underestimating OLR compared to satellite data but overall shows a good performance in reproducing cold clouds \citep{Cho2020}. The differences between MP schemes and how they perform in different events suggest that, even though precipitation and OLR are connected, we cannot use only OLR to understand precipitation formation processes and many other factors need to be analysed, such as clouds phases and properties. \vskip 0.2cm

In the MCS case, our results show that CAPE is at its highest a couple of days before precipitation peak and at its lowest during the peak (Fig.\ref{cape}). This indicates that there is an accumulation of energy prior to the precipitation peak that gets released during the main precipitation event. Schemes that report relatively high CAPE values, namely Thompson and Morrison, also simulate the most intense precipitation peaks; while when CAPE is relatively low also the precipitation peak is, such as in WDM7 and WDM6. Spatially, CAPE patterns are heavily influenced by topography and low values broadly correspond to the areas of precipitation maximum. In the heavy snow system case, CAPE values are about ten times smaller and do not exhibit a clear relationship with precipitation (Fig.\ref{Scape}). A study over Southeast India shows results not in agreement with what we observed for the MCS \citep{ReshmiMohan2018}. In fact, their analysis suggests that CAPE reaches its highest values during precipitation peak. Moreover, in their study Thompson and Morrison display radically different trends while in our results the two MP schemes simulate similar values. This contrast could be caused by the domain taken into consideration to average out the values and the nature of the events: while we considered a MCS, in their study heavy precipitation was caused by monsoon season. Therefore, the cloud MP involved and the dominating precipitation formation processes are different and schemes could be more or less able to reproduce the MP of one of the two cases. On the other hand, another research performed on an extreme summer precipitation event over Southwest India reports that maximum CAPE values were reached prior to the precipitation peak \citep{Thomas2021}, thus in agreement with our findings. Multiple studies concluded that Thompson tends to produce stronger warming which corresponds to higher CAPE values and stronger vertical updrafts \citep{Podeti2020,Thomas2021}.  Despite contrasting results, it is certain that the variation of CAPE values among MP schemes indicates that they simulate different instabilities of the atmosphere, which in turn affects the size of hydrometeors and precipitation mechanisms by altering vertical motions.

\subsection{Moisture}%2 pag
Understanding the intricate relationship between moisture flux and precipitation is crucial and yet remains a challenge due to inadequate observational data and complex terrain \citep{Lv2020}. In both case studies, our results for moisture flux at the surface show that the mean flux is always directed upward and diminishes when precipitation is more intense (Fig.\ref{qflux}, Fig.\ref{Sqflux}). Moreover, the association found between topography and moisture flux is probably linked to surface temperature, which is on average higher at lower altitudes, and its influence on evapotranspiration processes. In the MCS case, there seems to be no simple relationship between the intensity of moisture flux at surface and precipitation across MP schemes, because relatively lower moisture flux values in one MP scheme do not necessarily correspond to relatively higher precipitation. However, in the heavy snow system case, MP schemes that resemble the most ERA5 - WDM7 and WDM6 - are also the ones that perform better at precipitation simulation, suggesting that in this event precipitation is fueled by low-level moisture. A research on a heavy precipitation event over the Yangtze river basin revealed that precipitation was governed by an eastward moving cloud system sustained by low-level water vapor \citep{Chen2020}. A simulation study carried out for two summer months over central TP suggests that Thompson performs well in reproducing local precipitation and moisture processes \citep{Lv2020}. Since precipitation generation is primarily triggered by moisture advection, uplift and environmental instability \citep{Norris2015}, it would be useful to analyse the moisture transport at multiple atmospheric levels and establish the main mechanisms contributing to moisture supply.  \vskip 0.2cm

The vertical profile of Q reveals virtually null values above 200 hPa and a rapid increase going towards the surface (Fig.\ref{WTHQ}, Fig.\ref{SWTHQ}). In both study cases, ERA5 and WRF simulations diverge in the lower levels of atmosphere. The discrepancies in Q vertical profiles close to the surface between MP schemes likely cause the differences in accumulated precipitation at the end of the events, as MP schemes with lower Q also exhibit lower accumulated precipitation (Fig.\ref{PPOLRT}d, Fig.\ref{SPPOLRT}d). However, this is not valid for ERA5, that produces lower values of Q close to the surface but does not underestimate the overall accumulated precipitation. In fact, Q influence on precipitation also depends on its  vertical distribution and it is thus related to its transport, especially in the lower troposphere \citep{Zhou2024}. Close to the surface, ERA5 moisture transport upward is stronger than that of WRF simulations (Fig.\ref{wq}, Fig.\ref{Swq}). Thus, a more efficient moisture transport to higher levels could explain why ERA5 accumulated precipitation is not underestimated, because more moisture reaching higher levels of atmosphere leads to more possibility of cloud condensation and thus precipitation formation. A previous study in the Langtang Valley, Himalaya, reported large differences in the horizontal moisture transport between Thompson, Morrison and WDM6 schemes, which impacted cloud formation and precipitation outcome \citep{Orr2017}. Through testing WRF runs against validation datasets, they concluded that Morrison exhibited the best performance. A sensitivity study of summer 2018 conducted over the TP suggests that the vertically integrated water vapor (VIWV) spatial distribution closely resembles that of precipitation in most MP schemes. Morrison and Ylin produce higher VIWV and precipitation compared to Thompson, particularly over the southern TP \citep{Zhou2024}. Our findings agree with previous researches, but also indicate that the performance of MP schemes depends on the event considered. In the MCS case study Morrison, Thompson and CAM produce simulations of Q that lead to similar accumulated precipitation to that of ERA5 and IMERG, while in the heavy snow case study their performances are the poorest and the best performing MP schemes are WDM6 and WDM7 instead. \vskip 0.2cm

Horizontal winds play a major role in the transport of moisture across regions. Our findings illustrate that there are major discrepancies in wind representation throughout the atmosphere (Fig.\ref{windslati}, Fig.\ref{Swindslat}). Specifically, in the MCS case WRF simulations exhibit major negative biases in winds directed towards north at 200 hPa and in winds directed southward at 800 hPa compared to ERA5. In the heavy snow system case, there are clear positive biases in winds directed towards north between 100-400 hPa as well as close to the surface. These differences can affect moisture transport and could be the reason behind the observed southward shift of WRF precipitation in the MCS case and northward shift of WRF precipitation in the heavy snow system case. Moreover, the contrast in wind direction and intensity at different atmospheric levels could disrupt the convective system evolution and thus limit precipitation formation, causing the overall weaker precipitation observed in WDM7 and WDM6 in the MCS case. In the heavy snow system case, the stronger northward winds of Thompson, Morrison and Ylin could also explain why these schemes simulate more precipitation in the northern areas of the domain. Multiple previous studies \citep{Zhou2024,Tiwari2018} together with our analysis prove that the simulation of moisture content and transport is still not totally coherent among MP schemes in the TP region. \vskip 0.2cm

\subsection{Convection}%2 pag
MP schemes affect W through latent heating of atmosphere, which leads to low-level convergence. Intense updrafts can carry liquid drops above the 0°C level where they freeze, grow in size and form graupel and hail, which can lead to strong precipitation \citep{Kumjian2012}.
In our MCS analysis, W shows the same pattern as precipitation. In fact, during precipitation peak W exhibits its highest values and MP schemes showing intense and short W peaks - Thompson, Morrison and Ylin - reflect the same trend in precipitation intensity (Fig.\ref{Wwrf}). These are also the schemes resembling the most ERA5. The rest - CAM, WDM7 and WDM6 - show weaker and more limited updraft, causing lower precipitation. In the heavy snow system case, the timing is coherent between datasets, but previous considerations about W intensity still apply. MP schemes that display weaker W peak, namely WDM7 and WDM6, produce less intense precipitation compared to the rest of MP schemes and in this event are the most similar to ERA5. The rest of the schemes - Thompson, Morrison and Ylin - show higher W and more precipitation. Thus, it is clear that stronger updrafts are synonym of stronger convection which produces more precipitation, while lower W is an indication of weaker convection leading to less precipitation. W seems to be an extremely relevant factor contributing to the differences in precipitation simulation among MP schemes. A study over southeast India shows that Morrison and Thompson produced the strongest updraft, but Morrison did not accurately reproduce timing \citep{ReshmiMohan2018}. Another research on a MCS over the Tropical Western pacific demonstrated that Thompson and Morrison simulations displayed overall differences, but their representation of vertical velocity was quite similar. It was found that the main source of uncertainty was linked with solid hydrometeors and related processes \citep{VanWeverberg2013}. To further investigate convection processes, it would be beneficial to study other variables influencing the updrafts, such as the vertical profile of latent heating and MP processes occurring in the clouds.\vskip 0.2cm

MP affects atmospheric warming and circulation by both thermodynamics and dynamics mechanisms. Diabatic heating produced in the low troposphere by evaporative cooling and diabatic warming at higher levels caused by freezing can alterate the vertical velocity \citep{ReshmiMohan2018}. TH can be analysed to acquire additional knowledge on the convective activity. Nonetheless, from our results no clear indication on the evolution of both precipitation systems can be derived from TH profile and time series (Fig.\ref{WTHQ}, Fig.\ref{SWTHQ}). In fact, it remains constant throughout both events and does not differ between datasets. It would be useful to analyse also the equivalent potential temperature (eTH), which is defined as TH but considering moist air. In literature, it has been shown that high values of eTH indicate high values of moist convection and display major differences based on the choice of MP scheme \citep{ReshmiMohan2018}. Moreover, high values of eTH have been defined among key indicators for MCS identification \citep{Shukla2022}. To further investigate the relationship between TH and precipitation, future research could analyse the original vertical levels produced by WRF and cut off levels above 200 or 400 hPa. \vskip 0.2cm

PBLH is another important parameter as it determines the vertical degree of turbulent mixing and strongly influences the evolution of convective activity \citep{Li2023}. Higher values of PBLH indicate stronger convective activity and more efficient vertical transport of aerosols and moisture \citep{Masrour2023}. In the MCS case, all WRF schemes produce a similar temporal evolution of PBLH, despite the absolute values greatly varying, while ERA5 exhibits a completely different time series (Fig.\ref{pblh}). In fact, PBLH in WRF runs is at its highest a couple of days before precipitation peak and at its lowest during the peak; whereas in ERA5 the variable reaches the maximum value during precipitation peak. Results from the heavy snow system case show that WRF simulated PBLH more coherently, with the same time evolution as ERA5 but with values substantially bigger (Fig.\ref{Spblh}). In both cases, values among WRF runs differs a lot, but it is not true that higher (lower) PBLH values correspond to higher (lower) precipitation. ERA5 is proven to be able to reproduce well PBLH spatial patterns and time evolution over China \citep{Li2023}. However, great uncertainties are still caused by the differences between PBLH definition methods. In fact, this is the likely reason behind the variations in PBLH values between ERA5 and WRF simulations \citep{Slaettberg2021,Hong2006}. An interesting point for future analysis would be to investigate the association between W and PBLH, which in our current results is not clear.


\subsection{Cloud composition}\label{cloudc} %2.5?
Studying the cloud phase and composition can be useful for deepening our knowledge of which precipitation formation processes are dominant as well as determine which factors are the most influencial for precipitation uncertainties. Our results for the MCS case suggest that CAM shows the best performance in reproducing Qcloud vertical profile compared to ERA5 (Fig.\ref{Qcloud}) and in all WRF simulations Qcloud time evolution is in agreement with that of precipitation. Regarding Qice simulations Morrison, WDM7 and WDM6 are the most similar to ERA5 while Thompson severely underestimates it (Fig.\ref{Qice}). In the heavy snow system case, Ylin produces the closest Qcloud values to that of ERA5 (Fig.\ref{SQcloud}) and WDM7, WDM6 are the best in simulating Qice (Fig.\ref{SQice}). In this case, Qice time evolution corresponds to precipitation timing in all WRF runs, even though Thompson still simulates extremely low values. Multiple researches are in agreement with our findings, confirming that Thompson and Morrison Qcloud representantion are found to be similar, whereas Thompson substantially underestimates Qice values \citep{Podeti2020,ReshmiMohan2018}. Moreover, from our results it emerges that liquid precipitation is in agreement with Qcloud intensity and timing, while snow seems to be in agreement with Qice. \vskip 0.2cm

We also analysed Qrain, Qsnow and Qgraupel (Fig.\ref{Qrain}, Fig.\ref{Qsnow}, Fig.\ref{Qgraupel} for the MCS and Fig.\ref{SQrain}, Fig.\ref{SQsnow}, Fig.\ref{SQgraupel} for the heavy snow system). It is relevant to note that Qsnow exhibits values bigger of about one order of magnitude compared to the other solid-phase hydrometeors, suggesting that the aggregation processes play a primary role in cloud formation and its alterations greatly affect precipitation outcome in both study cases, as found in previous research \citep{MartinezCastro2019}. A study over western Hymalayas further confirms our results, as it shows that Qsnow has the largest magnitude out of all the other hydrometeors \citep{Tiwari2018}. They also demonstrated that maximum precipitation coincides with maximum amount of rain, snow and graupel, and that Thompson produces higher Qsnow compared to Morrison. They identified the reason in the fact that in Thompson snow size distribution depends on ice water content and temperature, while in Morrison the snow is assumed spherical with constant density. Another study performed over the Langtang Valley, Hymalaya, seems to be a further confirmation of our study both for non-precipitable and precipitable water species \citep{Orr2017}. In a research about extreme summer precipitation over the Central Hymalayas, Thompson and Morrison once again are identified as the best performing MP schemes followed by CAM, which seems to have captured the main features of the precipitation event \citep{Karki2018}. Multiple previous studies are in line with our findings showing that hydrometeors vertical distributions, intensity and timing substantially affect the precipitation outcome in WRF simulations. In particular, Qsnow has been identified to play a major role in precipitation formation due to its large mixing ratio. \vskip 0.2cm

By analysing LWP and IWP, the accuracy of humidity and precipitation simulations can be improved \citep{Chen2015,Chen2016}. Generally, both LWP and IWP confirm what already has been observed with the hydrometeors but their analysis allows to gain more insights on the spatial patterns. In the MCS case, the temporal evolution of LWP is well reproduced by most of WRF sumulations, except CAM (Fig.\ref{LWP}). IWP is underestimated by half of WRF simulations and overestimated by the rest, with Morrison being the most similar to ERA5 (Fig.\ref{IWP}). Both LWP and IWP spatial patterns are very different among MP schemes, but they generally reproduce the spatial distributions of precipitation. In the heavy snow system case study, the temporal evolution and spatial patterns of LWP is widely different depending on MP schemes and they are all very different compared to ERA5 (Fig.\ref{SLWP}). Instead, IWP is coherently reproduced by all datasets and WDM7, WDM6 produce very similar values to ERA5 (Fig.\ref{SIWP}). The discrepancies among MP schemes could be due to other parameters involved in precipitation formation, such as conversion processes between hydrometeors and wind dispersal mechanisms. The complex relationship between precipitation and water paths can be further analysed by considering partial correlation coefficients (Table \ref{corrs}, Table \ref{Scorrs}). In the MCS case, Thompson, Morrison and Ylin produce high correlation coefficients between precipitation and LWP, whereas the remaining WRF runs and ERA5 display a greater correlation between precipitation and IWP. This indicates that in the first MP schemes, precipitation formation is majorly affected by liquid hydrometeors, while in the remaining MP schemes the dominant mechanisms are related to solid hydrometeors. On the contrary, in the heavy snow system case the correlation coefficients for all datasets are extremely high with IWP, indicating that solid-phase hydrometeors are by far the most important factors in snow formation processes. A sensitivity study carried out over the Hanjiang River Basin, demonstrated that precipitation during summer is very sensitive to solid-phase hydrometeors processes \citep{Yang2021}. Various studies \citep{Tang2019,MartinezCastro2019} also suggested that warm-rain processes have a minor role in the direct formation of precipitation, but they are essential for the occurrance of supercooled water droplets and graupel embryos, which in turn are vital for the formation of precipitation. The vast plethora of correlation values and hydrometeors distributions proves that the various MP schemes simulate cloud properties differently and are a major source of uncertainties. Our results suggest that in both study cases, LWP and IWP take part into precipitation formation processes but solid-phase hydrometeors seem to be dominant overall.

\subsection{Characteristics of microphysics schemes}
Overall, Thompson and Morrison give the best (worst) simulations of the MCS (heavy snow) event, while WDM7 and WDM6 are the best (worst) performing schemes for the heavy snow system (MCS). Nonetheless, the accuracy of the simulations of individual variables is still widely different depending on the MP scheme used. From our results, Thompson and Morrison seem to simulate more accurately mixed-phase could MP processes, while WDM7 and WDM6 better reproduce cold clouds. It has been shown in previous studies \citep{Lim2010} that WDM6 and Morrison simulations of a precipitation event led to substantial differences. The determination of what caused these discrepancies is highly complex, but the absence of enhanced melting processes of snow and graupel in Morrison could be a relevant contribution. Therefore, the differences in precipitation simulations in MP schemes could be due to variations in the descriptions of shape and distributions of hydrometeors, but they could also be related to the ability of the schemes to convert non-precipitable hydrometeors to precipitable ones. In fact, by comparing the magnitudes of water species in both case studies, it emerges that Thompson and Morrison convert more efficiently from non-precipitable to precipitable hydrometeors, while WDM7 and WDM6 are less efficient. This leads to higher simulated precipitation by the first two MP schemes and lower simulated precipitation by the latter two. The very similar results of WDM7 and WDM6 also prove that the inclusion of an additional water species - hail in WDM7 - does not necessarily improve precipitation simulation and the actual parametrisation of other factors, such as the shape of hydrometeors and conversion processes, might have as well a relevant impact. CAM and Ylin overall performances lie somewhere in between and produce distinct simulations compared to the other MP schemes. This could be due to the fact that they do not simulate Qgraupel but it could also be caused by other differences in the description of MP processes, such as particle size distributions. \\ \\

\section{Limitations and future research prospects}
The limitations of this project can be divided into two main types: computational limits and availability of observational data. The first type includes issues concerning the processing of large datasets with very high resolution. It was for this reason that, as previously discussed, we had to lower the spatial resolution of 3D variables. This inevitably impacts the amount of detail available in the results but does not jeopardise them. Moreover, we were not able to complete the simulation of the heavy snow system using CAM, so we analysed five MP schemes instead of six. The second type refers to the fact that observations were scarce and with relatively low resolution. ERA5 is an excellent product to test our simulations against, but acquiring additional ground- and satellite-based datasets would be a great asset. However, ground-based datasets are especially difficult to implement in the region studied because of the complex topography and harsh climate, while satellite-based products are available but their spatial domain and temporal resolution are heavily influenced by their orbits. \vskip 0.2cm
Our results could be expanded by including additional variables and relevant processes that have not been considered so far. For instance, analysing horizontal moisture transport and sources in the region could provide further information relatively to cloud formation and precipitation generation. The study of latent heating vertical profile and equivalent potential temperature could also give additional insight on convection processes and hydrometeors autoconversion. Lastly, it would be greatly beneficial to our study to analyse more in dept hydrometeors mechanisms that could influence precipitation outcome, such as ice growth, supercooled droplets and autoconversion processes between water species.



\chapter{Summary and conclusions}
%\addcontentsline{toc}{chapter}{Conclusions}
\rhead{\textsl{Chapter 5: Summary and conclusions}}
\renewcommand{\headrulewidth}{0pt}
In this study, we used different MP schemes within the state-of-the-art WRF model to simulate two intense precipitation events over the TP: a summer MCS and a heavy snow system in early autumn. Firstly, we assessed the discrepancies in temporal and spatial patterns of precipitation due to the choice of MP scheme. Then, we investigated precipitation related atmospheric processes by studying the energy budget, atmospheric instability, moisture transport and hydrometeors distributions. Lastly we identified key factors leading to modelling uncertainties and their relationships with MP schemes characteristics. When possible, our simulations were tested against observations and reanalysis datasets. The key findings are the following:
\begin{enumerate}
\item \textbf{The choice of MP scheme leads to major differences in precipitation spatial distribution and intensity, which can cause variations in accumulated precipitation equal to $\sim$40\% of the total amount}. Moreover, in the MCS case WRF simulations anticipate the precipitation peak of about 1-2 days compared to validation datasets, while in the heavy snow case the timing is coherently simulated. In the first case study, Thompson and Morrison give the best performances while WDM7 and WDM6 display the biggest dicrepancies. On the contrary, in the second case study WDM7 and WDM6 are the best performing MP schemes. In the MCS case, there is a clear relationship between precipitation and OLR, which confirms that precipitation was caused by deep convection. %The total accumulated precipitation is underestimated by half of the schemes (Ylin, WDM7 and WDM6), while it is fairly reproduced by the rest (Thompson, Morrison and CAM). This could be explained by the vertical profile of specific humidity together with its upward transport.
\item \textbf{In the MCS case CAPE exhibits a connection with precipitation variations in timing and intensity}, and MP schemes with higher CAPE also show more intense precipitation. However, WRF runs simulate values two to three times lower than ERA5. In the heavy snow system case, CAPE values are roughly ten times smaller and do not exhibit any relationship with precipitation. 
\item The vertical profile and intensity of W are heavily dependent on the MP scheme used. It is in good agreement with precipitation patterns and our results suggest that \textbf{W could be a major cause of discrepancies in precipitation simulations} by affecting convection processes. Our findings show that Thompson and Morrison give the best performances in W simulation in the MCS case, while WDM7 and WDM6 are better in the heavy snow system case.
\item In both case studies, \textbf{solid-phase water species are identified as crucial factors in precipitation simulations variations}. It is clear that the representation of water species and their phases leads to major variations in precipitation simulations among MP schemes. In particular, Qsnow displays the biggest values out of all water species and its timing and intensity coincide with precipitation patterns.
\item Partial correlation results suggest that \textbf{the impact of liquid- and solid-phase hydrometeors on precipitation formation processes greatly varies depending on the MP scheme and the nature of the precipitation event}. In the MCS case, precipitation formation processes are dominated by liquid species in some of the MP schemes and by solid species in others. On the contrary, precipitation formation during the heavy snow system is mainly influenced by solid-phase hydrometeors. \vskip 0.2cm
\item Variations in wind intensity at multiple levels of atmosphere could disrupt the formation of the precipitation system, leading to weaker and more broadly distributed precipitation in CAM, WDM7 and WDM6. Moreover, \textbf{the difference in intensity of northward winds could explain the shift in precipitation simulation in WRF} compared to ERA5 both in the MCS and heavy snow system.
\item Our results show that \textbf{Thompson and Morrison are generally the best performing MP schemes in simulating the MCS, while WDM7 and WDM6 produce the best simulations of the heavy snow system}. CAM and Ylin overall performances lie somewhere in between. This could be due to the fact that they do not simulate Qgraupel but it could also be caused by other differences in the description of MP processes.
\end{enumerate} 







%%%%%%%%%%%%%%%%%%%%%%%%%%%%%%%%
\newpage
\rhead{\textsl{Bibliography}}
\renewcommand{\headrulewidth}{0pt}
%\bibliographystyle{apacite}
{\footnotesize\bibliography{microX} }
%\printbibliography



%%%%%%%%%%%%%%%%%%%%%%%%%%%%%%%%
\newpage
\pagenumbering{Roman}
\appendix
\rhead{\textsl{Appendix A: MCS}}
\renewcommand{\headrulewidth}{0pt}
\section*{Appendix A: MCS}
\renewcommand\thefigure{A\arabic{figure}} 

\begin{figure}[!htb]\centering\includegraphics[width=0.98\textwidth]{./resultsMCS/decycled}\caption{Temporal evolution and correlogram of T2m (left) and OLR (right) without diurnal cycle.} \label{decy} \end{figure}
\begin{figure}[htbp]\centering\includegraphics[width=0.85\textwidth]{./resultsMCS/Qhail}\caption{Qhail vertical profile temporal evolution (averaged over latitudes 26-35°N and longitudes 100-115°E) and vertical profile (averaged over July 16-24, latitudes 26-35°N and longitudes 100-115°E).} \label{Qhail}\end{figure}
\begin{figure}[!htb]\centering\includegraphics[width=0.98\textwidth]{./resultsMCS/winds_londiff}\caption{Vertical profile of eastward winds differences of WRF runs minus ERA5 averaged over latitudes (26-35°N) and time (July 16-24); and ERA5 vertical profile as reference.} \label{windslon}
\centering\includegraphics[width=0.98\textwidth]{./resultsMCS/WQ}\caption{Vertical profiles averaged over time (July 16-24), latitudes (26-35°N) and longitudes (100-115°E) of TH*W and W*Q.} \label{wq}\end{figure}

\pagebreak
\newpage
\clearpage
\rhead{\textsl{Appendix B: SNOW}}
\renewcommand{\headrulewidth}{0pt}
\section*{Appendix B: SNOW}
\setcounter{figure}{0} 
\renewcommand\thefigure{B\arabic{figure}} 
\begin{figure}[!htb]\centering\includegraphics[width=0.98\textwidth]{./resultsSNOW/decycled}\caption{Temporal evolution and correlogram of T2m (left) and OLR (right) without diurnal cycle.} \label{decy2} \end{figure}
\begin{figure}[htbp]\centering\includegraphics[width=0.85\textwidth]{./resultsSNOW/Qhail}\caption{Qhail vertical profile temporal evolution (averaged over latitudes 27-35°N and longitudes 90-105°E) and vertical profile (averaged over October 4-9, latitudes 27-35°N and longitudes 90-105°E).} \label{SQhail}\end{figure}
\begin{figure}[!htb]\centering\includegraphics[width=0.98\textwidth]{./resultsSNOW/windslong}\caption{Vertical profile of eastward winds differences of WRF runs minus ERA5 averaged over latitudes (27-35°N) and time (October 4-9); and ERA5 vertical profile as reference.} \label{Swindslon}
\centering\includegraphics[width=0.98\textwidth]{./resultsSNOW/WQ}\caption{Vertical profiles averaged over time (October 4-9), latitudes (27-35°N) and longitudes (90-105°E) of TH*W and W*Q.} \label{Swq}\end{figure}




\end{document}

































